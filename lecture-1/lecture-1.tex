\documentclass{beamer}
\usetheme{Frankfurt}

\usepackage{caption}
\usepackage{subcaption}
\usepackage{listings}
\usepackage{color}

\definecolor{thegreen}{rgb}{0,.5,0}
\definecolor{theblue}{rgb}{0,0,.8}
\lstset{% use our version of highlighting
  language=python,      % using python
  keywordstyle={\color{theblue}\bfseries},          % keywords
  commentstyle=\color{thegreen},         % comments
}
\lstset{%
  basicstyle={\ttfamily\normalsize},  % use font and smaller size
  basewidth={0.5em,0.5em},
  showstringspaces=false,                   % do not emphasize spaces in strings
  tabsize=2,                                % number of spaces of a TAB
  aboveskip={0\baselineskip},               % a bit of space above
  columns=fixed,                            % nice spacing
}

\AtBeginSection[]
{
  \begin{frame}<beamer>
    \frametitle{\thesection}
    \tableofcontents[currentsection]
  \end{frame}
}

\title{Introduction to Scientific Python \\
Lecture 1: Introduction to computing}
\author{Austin Benson \\
\vspace{0.1in}
Dan Frank \\
\vspace{0.1in}
Institue for Computational and Mathematical Engineering (ICME)}
\begin{document}
\maketitle

%\section{Matrix storage}

\begin{frame}
\frametitle{Course structure}

\begin{itemize}
\setlength{\itemsep}{0.2in}
\item{
Five 2-hour lectures, all in first three weeks
}

\item{
% this may change, leaving here as a filler
Homework due at the beginning of each lecture
}

\item{
Lectures:
\begin{itemize}
\setlength{\itemsep}{0.05in}
\item{1 hour of instruction and demos}
\item{1 hour of interactive exercises}
\end{itemize}
}

\item{
Please bring a laptop to class!
}
\end{itemize}

\vspace{0.2in}

All course materials on the course web site: \url{http://math.stanford.edu/~arbenson/cme193.html}

\end{frame}

\begin{frame}
\frametitle{Course outline}

\begin{enumerate}
\setlength{\itemsep}{0.2in}

\item{Introduction to Computing}
\item{Introduction to Python (Part I)}
\item{Introduction to Python (Part II)}
\item{Learning scientific tools}
\item{Using scientific tools}

\end{enumerate}

\end{frame}


\begin{frame}
\frametitle{About the instructions}

\begin{figure}
        \centering
	\begin{subfigure}[b]{0.45\textwidth}
	\centering
	\includegraphics[height=1in]{"images/austin"}
	\caption{Austin}
	\label{fig:hw2_15a}
	\end{subfigure}
        ~ %add desired spacing between images, e. g. ~, \quad, \qquad etc. 
          %(or a blank line to force the subfigure onto a new line)
	\begin{subfigure}[b]{0.45\textwidth}
	\centering
	\includegraphics[height=1in]{"images/dan"}
	\caption{Dan}
	\label{fig:hw2_15b}
	\end{subfigure}
	% \caption{}\label{fig:hw2_34}
\end{figure}

% office hours??
ICME PhD students

Offices: ICME (Huang basement)

\end{frame}

\section{What is a variable?}
\begin{frame}
\frametitle{Basic variables}

A variable holds information (1.343, 'hi', [1, 1, 2, 3, 5, 8])

\vspace{0.2in}

In Python this is simple:

\begin{center}
\begin{tabular}{c}
\lstinputlisting{code/basic_vars1.py}
\end{tabular}
\end{center}

(\textcolor{thegreen}{\#} signifies the start of a comment in Python)

\end{frame}


\begin{frame}
\frametitle{Basic variables}

Variables can change (they are \emph{variable})

\begin{center}
\begin{tabular}{c}
\lstinputlisting{code/basic_vars2.py}
\end{tabular}
\end{center}

(no more knowledge of 1.343, 2, or 14)

\end{frame}


\section{Arithmetic and boolean operators}

\begin{frame}
\frametitle{Arithmetic operators}

\begin{center}
\begin{tabular}{c}
\lstinputlisting{code/arith_ops1.py}
\end{tabular}
\end{center}

Python supports this ``arithmetic" on strings.  (Note: Matlab does not.)
\end{frame}

\begin{frame}
\frametitle{Arithmetic operators}

Some nice shortcuts:

\begin{center}
\begin{tabular}{c}
\lstinputlisting{code/arith_ops2.py}
\end{tabular}
\end{center}

\end{frame}


\begin{frame}
\frametitle{Comparison operators}

\begin{center}
\begin{tabular}{c}
\lstinputlisting{code/comp_ops1.py}
\end{tabular}
\end{center}

\end{frame}

\begin{frame}
\frametitle{Boolean operations}

\begin{center}
\begin{tabular}{c}
\lstinputlisting{code/bool_ops1.py}
\end{tabular}
\end{center}

Python conveniently uses the keywords \textcolor{theblue}{and}, \textcolor{theblue}{or}, \textcolor{theblue}{not} instead of the symbols \textcolor{theblue}{\&\&}, $\textcolor{theblue}{||}$, \textcolor{theblue}{!}

\end{frame}


\section{Control Flow}
\begin{frame}
\frametitle{if}
The \texttt{if} statement is the most basic way to control the direction and flow of a program

\begin{center}
\begin{tabular}{c}
\lstinputlisting{code/if1.py}
\end{tabular}
\end{center}

\end{frame}


\begin{frame}
\frametitle{if/else}
\texttt{if} is often accompanied by \texttt{else} to specify a second path for the code if the \texttt{if} statement is false

\begin{center}
\begin{tabular}{c}
\lstinputlisting{code/if2.py}
\end{tabular}
\end{center}

\end{frame}

% TODO: maybe a section on if/elif

\begin{frame}
\frametitle{while}
The \texttt{while} loop repeatedly executes a task while a condition is true

\begin{center}
\begin{tabular}{c}
\lstinputlisting{code/while1.py}
\end{tabular}
\end{center}

What is this code doing?

\end{frame}

\begin{frame}
\frametitle{for}
The \texttt{for} loop also repeatedly executes a task

\vspace{0.1in}

Typically, \texttt{for} provides more structure:

\vspace{0.1in}

\centering
for ($i = 0$; $i < n$; $i = i + 1$)

\end{frame}

\begin{frame}
\frametitle{for}

%\begin{figure}[h!]
\centering
\includegraphics[height=1.5in]{"images/for"}
%\caption{}
%\label{}
%\end{figure}
\end{frame}

\begin{frame}
\frametitle{Python for}
\begin{itemize}
\setlength{\itemsep}{0.2in}
\item{Python uses \texttt{for} differently than Matlab, C++, ...}
\item{\texttt{for} is used to iterate over elements in an ``object"}
\item{This is a reason why Python is easy and powerful}
\item{More next lecture on how you can iterate over a ``object"}
\end{itemize}

\end{frame}

\begin{frame}
\frametitle{Python for}

\begin{center}
\begin{tabular}{c}
\lstinputlisting{code/for1.py}
\end{tabular}
\end{center}

\end{frame}


\section{Functions}
\begin{frame}
\frametitle{Functions}
Functions are used to organize programs into coherent pieces

\vspace{0.2in}

In other words, functions are used to generalize or ``abstract" components of a program 
\end{frame}


\begin{frame}
\frametitle{Root finding}

What is a problem with this code?

\begin{center}
\begin{tabular}{c}
\lstinputlisting{code/while1.py}
\end{tabular}
\end{center}

\end{frame}

\begin{frame}
\frametitle{Root finding}
We do not want a $\sqrt{7}$ method, we want a $\sqrt{x}$ method

\vspace{0.2in}

We still have the same capabilities (just let $x = 7$), but now our ``abstracted" root finder can be used for more cases
\end{frame}


\begin{frame}
\frametitle{Root finding}

With the function, we can call \texttt{root(7)}

\begin{center}
\begin{tabular}{c}
\lstinputlisting{code/while2.py}
\end{tabular}
\end{center}

\end{frame}

\begin{frame}
\frametitle{Root finding}
More general, we can call \texttt{root(7, 0.1)}:

\begin{center}
\begin{tabular}{c}
\lstinputlisting{code/while3.py}
\end{tabular}
\end{center}

\end{frame}


\begin{frame}
\frametitle{Root finding}
Python makes it easy to provide default argument values:

\begin{center}
\begin{tabular}{c}
\lstinputlisting{code/while4.py}
\end{tabular}
\end{center}

Can call \texttt{root(3)}, \texttt{root(113, 0.01)}
\end{frame}

\begin{frame}
\frametitle{Root finding}
A few things to think about:
\vspace{0.1in}
\begin{itemize}
\setlength{\itemsep}{0.1in}
\item{Are there cases where \texttt{root()} will have an error?}
\item{Are there cases where \texttt{root()} will run forever? (fail to converge)}
\item{How else can we generalize the function?}
\end{itemize}
\end{frame}


\begin{frame}
\frametitle{Basic functions}

What will happen when this code runs?

\begin{center}
\begin{tabular}{c}
\lstinputlisting{code/func1.py}
\end{tabular}
\end{center}

\end{frame}

\begin{frame}
\frametitle{Basic functions}

What about this code?

\begin{center}
\begin{tabular}{c}
\lstinputlisting{code/func2.py}
\end{tabular}
\end{center}

\end{frame}


\begin{frame}
\frametitle{More function examples}

\begin{center}
\begin{tabular}{c}
\lstinputlisting{code/poly_eval1.py}
\end{tabular}
\end{center}

\end{frame}

\begin{frame}
\frametitle{More function examples}

\begin{center}
\begin{tabular}{c}
\lstinputlisting{code/poly_eval2.py}
\end{tabular}
\end{center}

\end{frame}


\end{document}

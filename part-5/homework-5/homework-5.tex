%
\documentclass{article}
\newcommand{\assgnnum}{5 Extra Credit}
\newcommand{\duedate}{February 7}

\usepackage{amsmath}
\usepackage{fullpage}
\usepackage{amssymb}
%\usepackage{bbm}
\usepackage{fancyhdr}
%\usepackage{paralist}
\usepackage{graphicx}
\usepackage{caption}
\usepackage{subcaption}
\usepackage[pdftex,colorlinks=true, urlcolor = blue]{hyperref}
\usepackage{../../arbenson-math}


\oddsidemargin 0in \evensidemargin 0in
\topmargin -0.5in \headheight 0.25in \headsep 0.25in
\textwidth 6.5in \textheight 9in
\parskip 6pt \parindent 0in \footskip 20pt

% set the header up
\fancyhead{}
\fancyhead[L]{CME193: Assignment \assgnnum}
\fancyhead[R]{Due: \duedate}
%%%%%%%%%%%%%%%%%%%%%%%%%%
\renewcommand\headrulewidth{0.4pt}
\setlength\headheight{15pt}

\newcommand{\p}{\ensuremath{\mathbf{P}}}
\renewcommand{\Pr}[1]{\ensuremath{\p \left \{ #1 \right \}}}
\newcommand{\nti}{\ensuremath{n \to \infty}}
\newcommand{\I}{\ensuremath{\operatorname{I}}}
\newcommand{\One}[1]{\ensuremath{\mathbbm{1}_{\left \{ #1 \right \}}}}
\newcommand{\E}{\ensuremath{\mathbf{E}}}
\newcommand{\Ex}[2][]{\ensuremath{\E_{#1} \left[ #2 \right]}}
\newcommand{\var}{\ensuremath{\operatorname{Var}}}
\newcommand{\cov}{\ensuremath{\operatorname{Cov}}}
\newcommand{\F}{\ensuremath{\mathcal{F}}}
\newcommand{\R}{\ensuremath{\mathbb{R}}}
\newcommand{\C}{\ensuremath{\mathbb{C}}}
\newcommand{\NormRV}[2]{\ensuremath{\operatorname{N}\left(#1, #2\right)}}
\newcommand{\BetaRV}[2]{\ensuremath{\operatorname{Beta}\left(#1, #2\right)}}
\newcommand{\argmax}{\operatornamewithlimits{argmax}}
\newcommand{\x}{\mathbf{x}}
\newcommand{\A}{\mathbf{A}}
\newcommand{\bb}{\mathbf{b}}

\newcounter{points}
\setcounter{points}{0}
\newcounter{bonuspoints}
\setcounter{bonuspoints}{0}

\newcommand\setpoints[1]{\addtocounter{points}{#1}(#1 points)}
\newcommand\setpoint{\addtocounter{points}{1}(1 point)}
\newcommand\setbonuspoints[1]{\addtocounter{bonuspoints}{#1}(#1 bonus points)}
\newcommand\setbonuspoint{\addtocounter{bonuspoints}{1}(1 bonus point)}

\newcommand\printpoints{Total number of points: \value{\thepoints}}

\newcommand{\eqD}{\ensuremath{\overset{\mathcal{D}}{=}}}

\setlength{\parindent}{0in}
\usepackage[normalem]{ulem}

\begin{document}

\pagestyle{fancy}
%\vspace*{15pt}

Each problem below is worth one failed assignment. 
\begin{enumerate}

\item \textbf{Non-Uniform Sampling with Replacement} \setbonuspoint 

Given a finite set $A = \{a_1, a_2, \dots, a_n \} $ and a vector of probabilities of the same size \newline $ \boldsymbol{p} = (p_1, p_2, \dots , p_n),$ a non-uniform sample of size $m$ with respect to probabilities $\boldsymbol p$ is a set
$S$ of size $m$ where for each $s$ in $S$, $\mathbb{P}[s=a_i] = p_i $. 

\vspace{.1in}

Write a function that accepts two arrays, corresponding to $A$ and $\boldsymbol p$ above and an integer $m$ and produces a non-uniform sample from set $A$ with respect to $\boldsymbol p$ of size $m$. Verify that your function by checking that for large $m$ the proportion of $S$ equaling $a_i$ is approximately $p_i$.



\item \textbf{Kernel Density Estimation in 2D} \setbonuspoint

 Create a function that will compute a two dimensional KDE given two arrays, $x$ and $y$ and a bandwidth. Use a bivariate gaussian kernel. The bandwidth in this case will be the common diagonal variance of the gaussian kernel.


 
 %pprittponnts.
\end{document} 

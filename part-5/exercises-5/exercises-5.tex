%
\documentclass{article}
\newcommand{\assgnnum}{5}
\newcommand{\assigndate}{January 24}

\usepackage{amsmath}
%\usepackage{fullpage}
\usepackage{amssymb}
%\usepackage{bbm}
\usepackage{fancyhdr}
%\usepackage{paralist}
\usepackage{graphicx}
\usepackage[pdftex,colorlinks=true, urlcolor = blue]{hyperref}
\usepackage{../../arbenson-math}

\oddsidemargin 0in \evensidemargin 0in
\topmargin -0.5in \headheight 0.25in \headsep 0.25in
\textwidth 6.5in \textheight 9in
\parskip 6pt \parindent 0in \footskip 20pt

% set the header up
\fancyhead{}
\fancyhead[L]{CME193: In-class exercises \assgnnum}
\fancyhead[R]{\assigndate}
%%%%%%%%%%%%%%%%%%%%%%%%%%
\renewcommand\headrulewidth{0.4pt}
\setlength\headheight{15pt}


\newcommand{\p}{\ensuremath{\mathbf{P}}}
\renewcommand{\Pr}[1]{\ensuremath{\p \left \{ #1 \right \}}}
\newcommand{\nti}{\ensuremath{n \to \infty}}
\newcommand{\I}{\ensuremath{\operatorname{I}}}
\newcommand{\One}[1]{\ensuremath{\mathbbm{1}_{\left \{ #1 \right \}}}}
\newcommand{\E}{\ensuremath{\mathbf{E}}}
\newcommand{\Ex}[2][]{\ensuremath{\E_{#1} \left[ #2 \right]}}
\newcommand{\var}{\ensuremath{\operatorname{Var}}}
\newcommand{\cov}{\ensuremath{\operatorname{Cov}}}
\newcommand{\F}{\ensuremath{\mathcal{F}}}
\newcommand{\R}{\ensuremath{\mathbb{R}}}
\newcommand{\C}{\ensuremath{\mathbb{C}}}
\newcommand{\NormRV}[2]{\ensuremath{\operatorname{N}\left(#1, #2\right)}}
\newcommand{\BetaRV}[2]{\ensuremath{\operatorname{Beta}\left(#1, #2\right)}}
\newcommand{\argmax}{\operatornamewithlimits{argmax}}
\newcommand{\x}{\mathbf{x}}
\newcommand{\A}{\mathbf{A}}
\newcommand{\bb}{\mathbf{b}}


\newcounter{points}
\setcounter{points}{0}

\newcommand\setpoints[1]{\addtocounter{points}{#1}(#1 points)}
\newcommand\printpoints{Total number of points: \thepoints}

\newcommand{\eqD}{\ensuremath{\overset{\mathcal{D}}{=}}}

\setlength{\parindent}{0in}

\begin{document}

\pagestyle{fancy}
%\vspace*{15pt}
\begin{enumerate}

\item \textbf{KDE Plot}

Recall the definition of a KDE: given a sample $x_1, x_2, \dots, x_n$ from an unknown distribution $f$ the kernel density estimate of $f$ at point $x$ with kernel $K_b$ is defined as 
$$\hat{f}(x;b) = \frac{1}{n}\sum_{i=1}^{n}K_b(x-x_i)$$
where the kernel must satisfy $\int_{-\infty}^{\infty}K(u)du = 1$ and $K(-u) = K(u).$ The parameter $b$ is known as the bandwidth and controls the width of the kernel used. 
\begin{enumerate}
\item generate data from the exponential distribution with scale 1 
\item write a function that computes a KDE with a Gaussian kernel. Here the bandwidth refers to the standard deviation of the Gaussian kernel.
\item plot both the data you generated as X's and the KDE as a line plot on the same axis
\end{enumerate}







\item \textbf{KDE Plot 2D}

\item \textbf{Simple Web Scrape}



\item \textbf{Array Creation and Operations}
\begin{enumerate}
\item 
Create the following array
$\begin{bmatrix}
1 & 1 & 1 & 1 & 1\\
1 & 2 & 1 & 1 & 1\\
1 & 1 & 3 & 1 & 1\\
1 & 1 & 1 & 4 & 1
\end{bmatrix}$
\item  Compute the row sums of the above matrix 
\item Compute the column sums of the above matrix
\item
Download and read into memory the matrix found below. Check that it is equal to the array you created above. 
\begin{center}
\url{http://stanford.edu/~arbenson/cme193/data/lec4_array.txt}
\end{center}
\end{enumerate}

\item{\textbf{Array Slicing and Indexing}}

Using the array above return the second and third rows and the columns containing an even number as a $2 \times  2$ array using... 
\begin{enumerate}
%\setcounter{enumii}{1}
\item integer indexes
\item slices
\item boolean arrays
\item boolean arrays computed from the array
\end{enumerate}

\item \textbf{Broadcasting}

Using the above array assigned as \textit{arr}, describe the following operations

\begin{enumerate}


\item 
\begin{verbatim}
arr * 5.
\end{verbatim}


\item 
\begin{verbatim}
arr * np.arange(arr.shape[1])
\end{verbatim}

\item 
\begin{verbatim}
arr * np.arange(arr.shape[0])
\end{verbatim}


\item 
\begin{verbatim}
arr.T * np.arange(arr.shape[0])
\end{verbatim}

\item compute the dot product of the array with $\begin{bmatrix}
0\\
1\\
2\\
3\\
4\\
\end{bmatrix}$
in two ways
\end{enumerate}
%\printpoints.

\end{enumerate}
\end{document} 

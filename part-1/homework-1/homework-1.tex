\documentclass{article}
\newcommand{\assgnnum}{1}
\newcommand{\duedate}{April 9}

\usepackage{amsmath}
%\usepackage{fullpage}
\usepackage{amssymb}
%\usepackage{bbm}
\usepackage{fancyhdr}
%\usepackage{paralist}
\usepackage{graphicx}
\usepackage[pdftex,colorlinks=true, urlcolor = blue]{hyperref}
\usepackage{listings}
\usepackage{color}
\usepackage{xcolor}

\definecolor{thegreen}{rgb}{0,.5,0}
\definecolor{comment-green}{rgb}{0,.3,0}
\definecolor{theblue}{rgb}{0,0,.8}
\definecolor{light-gray}{gray}{0.98}
\definecolor{comment-color}{rgb}{0,0,.8}
\definecolor{string-color}{rgb}{0,.75,0}
\definecolor{border-blue}{rgb}{0,0,.6}

\lstset{% use our version of highlighting                                            
  language=python,      % using python                                               
  keywordstyle={\color{teal}\bfseries},          % keywords                          
  commentstyle=\color{comment-color},         % comments                             
  stringstyle=\color{string-color},                   %strings                       
}

\lstset{
  basicstyle={\ttfamily\normalsize},  % use font and smaller size                    
  basewidth={0.5em,0.5em},
  showstringspaces=false,                   % do not emphasize spaces in strings     
  tabsize=2,                                % number of spaces of a TAB              
  aboveskip={0\baselineskip},               % a bit of space above                   
  columns=fixed,                            % nice spacing                           
}


\oddsidemargin 0in \evensidemargin 0in
\topmargin -0.5in \headheight 0.25in \headsep 0.25in
\textwidth 6.5in \textheight 9in
\parskip 6pt \parindent 0in \footskip 20pt

% set the header up
\fancyhead{}
\fancyhead[L]{CME193: Assignment \assgnnum}
\fancyhead[R]{Due: \duedate}
%%%%%%%%%%%%%%%%%%%%%%%%%%
\renewcommand\headrulewidth{0.4pt}
\setlength\headheight{15pt}

\newcommand{\p}{\ensuremath{\mathbf{P}}}
\renewcommand{\Pr}[1]{\ensuremath{\p \left \{ #1 \right \}}}
\newcommand{\nti}{\ensuremath{n \to \infty}}
\newcommand{\I}{\ensuremath{\operatorname{I}}}
\newcommand{\One}[1]{\ensuremath{\mathbbm{1}_{\left \{ #1 \right \}}}}
\newcommand{\E}{\ensuremath{\mathbf{E}}}
\newcommand{\Ex}[2][]{\ensuremath{\E_{#1} \left[ #2 \right]}}
\newcommand{\var}{\ensuremath{\operatorname{Var}}}
\newcommand{\cov}{\ensuremath{\operatorname{Cov}}}
\newcommand{\F}{\ensuremath{\mathcal{F}}}
\newcommand{\R}{\ensuremath{\mathbb{R}}}
\newcommand{\C}{\ensuremath{\mathbb{C}}}
\newcommand{\NormRV}[2]{\ensuremath{\operatorname{N}\left(#1, #2\right)}}
\newcommand{\BetaRV}[2]{\ensuremath{\operatorname{Beta}\left(#1, #2\right)}}
\newcommand{\argmax}{\operatornamewithlimits{argmax}}
\newcommand{\x}{\mathbf{x}}
\newcommand{\A}{\mathbf{A}}
\newcommand{\bb}{\mathbf{b}}

\newcounter{points}
\setcounter{points}{0}
\newcounter{bonuspoints}
\setcounter{bonuspoints}{0}

\newcommand\setpoints[1]{\addtocounter{points}{#1}(#1 points)}
\newcommand\setpoint{\addtocounter{points}{1}(1 point)}
\newcommand\setbonuspoints[1]{\addtocounter{bonuspoints}{#1}(#1 bonus points)}
\newcommand\setbonuspoint{\addtocounter{bonuspoints}{1}(1 bonus point)}

\newcommand\printpoints{Total number of points: \value{\thepoints}}

\newcommand{\eqD}{\ensuremath{\overset{\mathcal{D}}{=}}}

\setlength{\parindent}{0in}

\begin{document}

\pagestyle{fancy}
%\vspace*{15pt}

There are 30 total points.  70+\% correctness (21+ points) is needed to pass.  Remember that you must pass all assignments to pass the class.  The assignment is due at the beginning of class.

This document is available at \url{http://stanford.edu/~arbenson/cme193/homeworks/hw-1.pdf} and on Coursework.  The starter code is available at \url{http://stanford.edu/~arbenson/cme193/homeworks/code-hw-1.zip} and on Coursework.

\vspace{0.3in}

\textbf{Overview and grading}

In this assignment, you are provided with three python files, and you are asked to implement functions in these files.  The functions are designed to give you practice with control flow and basic arithmetic in Python.  You will have to edit the Python files using a text editor.  You are welcome to use an integrated development environment (IDE), for example, IPython or Eclipse with the PyDev add-on.

To grade this assignment, tests will be conducted by calling the functions you implement.  Each test is worth 1 point and there is no partial credit.  There are 30 total tests.

All of the tests used for grading are distributed with the assignment.  Gaming the autograder by hard-coding the answers of the provided test functions is considered cheating and a violation of the Stanford honor code.

To run the autograder, place the file \texttt{grader.py} in the same directory as \texttt{arith.py}, \texttt{str.py}, and \texttt{simple\_math.py}.  You can run the autograder with the following command:
\begin{center}
\texttt{python grader.py}
\end{center}

You will need to be able to run Python on your computer in order to run the autograder.  There are instructions on the course web site.  You can also use the Stanford corn machines at \texttt{corn.stanford.edu}.

\vspace{0.2in}

\begin{enumerate}

\item \textbf{Preliminaries} \\
 Make sure that you understand the concepts from Homework 0. \setpoints{0}.

%\end{enumerate}

%\begin{enumerate}
%\setcounter{enumii}{1}
%\item In the following questions, you will have to implement some Python functions.  Python uses white space as indentation to group statements.  You will have to understand this to implement functions.  The following resources may be helpful:

%\begin{enumerate}
%\item \url{http://docs.python.org/2/reference/lexical_analysis.html}, Sections 2.1.1-2.1.3, 2.1.5-2.1.9, 2.3.1.  This section of the Python docs contains vocabulary from Programming Language theory, so do not worry if you do not understand everything.  However, the examples on indentation and logical lines are useful.
%\item \url{}
%\end{enumerate}

%\end{enumerate}



\item \textbf{Arithmetic} \\
Implement the functions \texttt{arith2()}, \texttt{arith3()}, and \texttt{arith4()} in the file \texttt{arith.py}.  As an example, the function \texttt{arith1()} has already been implemented for you.  \setpoints{14}.

\item \textbf{Strings} \\
Implement the functions \texttt{str2()}, \texttt{str3()}, and \texttt{str4()} in the file \texttt{strings.py}.  As an example, the function \texttt{str1()} has already been implemented for you. Hint: the \texttt{len()} function may be useful for implementing \texttt{str4()}.  See: \url{http://docs.python.org/2/library/functions.html#len}. \setpoints{8}.  

\item \textbf{Simple math functions} \\
Implement the functions \texttt{seq\_add()} and \texttt{fact()} in the file \texttt{simple\_math.py}.  You are not allowed to use the Python math library. \setpoints{8}.

\item \textbf{Submission} \\
Upload the files \texttt{arith.py}, \texttt{strings.py}, and \texttt{simple\_math.py} to your Drop Box on coursework.
\end{enumerate}

\end{document} 

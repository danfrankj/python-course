\documentclass{beamer}
\usepackage{caption}
\usepackage{subcaption}
\usepackage{../../arbenson-math}

\usetheme{boxes}
\usecolortheme{seahorse}

\AtBeginSection[]
{
  \begin{frame}<beamer>
    \frametitle{\thesection}
    \tableofcontents[currentsection]
  \end{frame}
}

\title{CME 193: Introduction to Scientific Python \\
Lecture 1: Introduction to computing}
\author{Austin Benson \\
\vspace{0.1in}
Dan Frank \\
\vspace{0.1in}
Institute for Computational and Mathematical Engineering (ICME)}
\begin{document}
\maketitle

\section{Administrivia}
\begin{frame}
\frametitle{About the instructors}

\begin{figure}
        \centering
	\begin{subfigure}[b]{0.45\textwidth}
	\centering
	\includegraphics[height=1in]{"images/austin"}
	\caption{Austin}
	\label{fig:hw2_15a}
	\end{subfigure}
        ~ %add desired spacing between images, e. g. ~, \quad, \qquad etc. 
          %(or a blank line to force the subfigure onto a new line)
	\begin{subfigure}[b]{0.45\textwidth}
	\centering
	\includegraphics[height=1in]{"images/dan"}
	\caption{Dan}
	\label{fig:hw2_15b}
	\end{subfigure}
	% \caption{}\label{fig:hw2_34}
\end{figure}

% office hours??
ICME PhD students

Offices: ICME (Huang basement)
\end{frame}

\begin{frame}
\frametitle{Why listen to us?}

\begin{itemize}
\setlength{\itemsep}{0.2in}
\item{Austin:
\begin{itemize}
\setlength{\itemsep}{0.05in}
\item{Python for research}
\item{Python at Google}
\item{Many interviews}
\end{itemize}
}
\item{Dan:
\begin{itemize}
\setlength{\itemsep}{0.05in}
\item{Python for research}
\item{Python at Twitter}
\end{itemize}
}
\end{itemize}

\end{frame}


\begin{frame}
\frametitle{Course structure}

\begin{itemize}
\setlength{\itemsep}{0.2in}
\item{
Five 2-hour lectures, all in first three weeks
}

\item{
% this may change, leaving here as a filler
Homework corresponding to each lecture (5 total)
%\begin{itemize}
%\setlength{\itemsep}{0.05in}
%\item{All electronic submissions}
%\end{itemize}
}

\item{
Lectures:
\begin{itemize}
\setlength{\itemsep}{0.05in}
\item{1 hour of instruction and demos}
\item{1 hour of interactive exercises}
\end{itemize}
}

\item{
Please bring a laptop to class!
}
\end{itemize}

\vspace{0.2in}

All course materials on the course web site: \url{http://www.stanford.edu/~arbenson/cme193.html}

\end{frame}

\begin{frame}
\frametitle{Office hours}

\begin{itemize}
\setlength{\itemsep}{0.2in}
\item{Austin: Monday and Wednesday, 4-5pm, by appointment}
\item{Extra: Friday 1/11 1-2pm (tomorrow)}
\item{Dan: TBA}
\item{Location: Huang basement near ICME}
\end{itemize}

\end{frame}


\begin{frame}
\frametitle{Passing the class}

The grading scheme is Satisfactory/Not Satisfactory.

\vspace{0.2in}

To pass, you need to complete \textbf{all} homework assignments and earn at least \textbf{70\%} of the points on \textbf{each} assignment.

\end{frame}


\begin{frame}
\frametitle{Course textbook}

\begin{itemize}
\setlength{\itemsep}{0.2in}
\item{No textbook!}
\item{We will post online materials on the course web site}
\end{itemize}

\end{frame}


\begin{frame}
\frametitle{Variety}
One reason why we are excited to teach this class:

\begin{itemize}
\item{Students in: CEE, Bioengr., Mech. Engr., Petroleum Engr., EE, ERE, ICME, MSE, Econ, Math, Stats, CS, Business, Biology, Biophysics, Biochemistry, Chemistry, Geophysics, Music, Political Science, Aero/Astro, Linguistics, Financial Math, Environmental Science, Undeclared}
\item{Undergraduate: year 1, 2, 3, 4; Graduate: year 1, 2, 3, 4+}
\end{itemize}

\end{frame}

\begin{frame}
\frametitle{Survey}

\begin{itemize}
\setlength{\itemsep}{0.2in}
\item{A lot of people use C++/Java}
\item{Problems: simulations, biology, stats/ML, web data, P = NP?}
\end{itemize}


\end{frame}


\begin{frame}
\frametitle{Course outline}

\begin{enumerate}
\setlength{\itemsep}{0.2in}

\item{Introduction to Computing}
\item{Data Structures}
\item{File I/O and Classes}
\item{Learning scientific tools}
\item{Using scientific tools}

\end{enumerate}

\end{frame}



\section{What is a variable?}
\begin{frame}
\frametitle{Basic variables}

A variable holds information (1.343, 'hi', [1, 1, 2, 3, 5, 8])

\vspace{0.2in}

In Python this is simple:

\codeblock{code/basic_vars1.py}

\textcolor{comment-color}{\#} signifies the start of a comment in Python.  The comment terminates at the end of the line.

\end{frame}


\begin{frame}
\frametitle{Basic variables}

Variables can change (they are \emph{variable})

\codeblock{code/basic_vars2.py}


(no more knowledge of 1.343, 2, or 14)

\end{frame}

\begin{frame}
\frametitle{Types}
Unlike C/C++ and Java, variables can change types. Python keeps track of the type internally.

\codeblock{code/basic_vars3.py}

If you have PL theory background: Python is strongly typed but not statically typed

\end{frame}


\section{Arithmetic and boolean operators}

\begin{frame}
\frametitle{Arithmetic operators}

\codeblock{code/arith_ops1.py}


Python supports this arithmetic on strings.  (Note: Matlab does not.)
\end{frame}

\begin{frame}
\frametitle{Arithmetic operators}

Shorthand to combine assignment and addition statements:

\codeblock{code/arith_ops2.py}


\vspace{0.1in}

\texttt{x} is now \texttt{6} and \texttt{y} is now \textcolor{string-color}{\texttt{'hihihihi'}}

\end{frame}


\begin{frame}
\frametitle{Comparison operators}

\codeblock{code/comp_ops1.py}


\end{frame}

\begin{frame}
\frametitle{Boolean operations}

\codeblock{code/bool_ops1.py}


Python conveniently uses the keywords \textcolor{teal}{and}, \textcolor{teal}{or}, \textcolor{teal}{not} instead of the symbols \textcolor{teal}{\&\&}, $\textcolor{teal}{||}$, \textcolor{teal}{!}

\end{frame}

\section{Control Flow}
\begin{frame}
\frametitle{if}
The \texttt{if} statement is the most basic way to control the direction and flow of a program

\codeblock{code/if1.py}


\end{frame}


\begin{frame}
\frametitle{if/else}
\texttt{if} is often accompanied by \texttt{else} to specify a second path for the code if the \texttt{if} statement is false

\codeblock{code/if2.py}


\end{frame}

\begin{frame}
\frametitle{Evaluating numerical values}

As boolean values, the numerical value 0 is False and all other numerical values are True (1, 4.33, -12, ...).

\codeblock{code/bool_ops2.py}


\end{frame}

\begin{frame}
\frametitle{while}
The \texttt{while} loop repeatedly executes a task while a condition is true

\codeblock{code/while1.py}


What is this code doing?

\end{frame}

\begin{frame}
\frametitle{for}
The \texttt{for} loop also repeatedly executes a task

\vspace{0.1in}

Typically, \texttt{for} provides more structure:

\vspace{0.1in}

\centering
for ($i = 0$; $i < n$; $i = i + 1$)

\end{frame}

\begin{frame}
\frametitle{for}

%\begin{figure}[h!]
\centering
\includegraphics[height=1.5in]{"images/for"}
%\caption{}
%\label{}
%\end{figure}
\end{frame}

\begin{frame}
\frametitle{Python for}
\begin{itemize}
\setlength{\itemsep}{0.2in}
\item{Python uses \texttt{for} differently than Matlab, C++, ...}
\item{\texttt{for} is used to iterate over elements in an ``object"}
\item{This is one reason why Python is easy and powerful}
\item{More next lecture on how you can iterate over a ``object"}
\end{itemize}

\end{frame}

\begin{frame}
\frametitle{Python for}

\codeblock{code/for1.py}


\end{frame}


\section{Functions}
\begin{frame}
\frametitle{Functions}
Functions are used to organize programs into coherent pieces

\vspace{0.2in}

In other words, functions are used to generalize or ``abstract" components of a program 
\end{frame}


\begin{frame}
\frametitle{Root finding}

What is a problem with this code?

\codeblock{code/while1.py}


\end{frame}

\begin{frame}
\frametitle{Root finding}
We do not want a $\sqrt{7}$ method, we want a $\sqrt{x}$ method

\vspace{0.2in}

We still have the same capabilities (just let $x = 7$), but now our ``abstracted" root finder can be used for more cases
\end{frame}


\begin{frame}
\frametitle{Root finding}

With this function, we can call \texttt{root(7)}

\codeblock{code/while2.py}


\end{frame}

\begin{frame}
\frametitle{Root finding}
More general, we can call \texttt{root(7, 0.1)}:

\codeblock{code/while3.py}


\end{frame}


\begin{frame}
\frametitle{Root finding}
Python makes it easy to provide default argument values:

\codeblock{code/while4.py}


Can call \texttt{root(3)}, \texttt{root(113, 0.01)}
\end{frame}

\begin{frame}
\frametitle{Root finding}
A few things to think about:
\vspace{0.1in}
\begin{itemize}
\setlength{\itemsep}{0.1in}
\item{Are there cases where \texttt{root()} will have an error?}
\item{Are there cases where \texttt{root()} will run forever? (fail to converge)}
\item{How else can we generalize the function?}
\end{itemize}
\end{frame}


\begin{frame}
\frametitle{Basic functions}

What will happen when this code runs?

\codeblock{code/func1.py}


\end{frame}

\begin{frame}
\frametitle{Basic functions}

What about this code?

\codeblock{code/func2.py}
\end{frame}


\begin{frame}
\frametitle{More function examples}

\codeblock{code/poly_eval1.py}
\end{frame}

\begin{frame}
\frametitle{More function examples}

\codeblock{code/poly_eval2.py}
\end{frame}


\begin{frame}
\frametitle{End}
\begin{itemize}
\setlength{\itemsep}{0.05in}
\item{Assignment 1 is posted on the course web site (due Tuesday 1/15)}
\item{In-class exercises, python codes from slides, and readings also posted}
\end{itemize}

Next time:
\begin{enumerate}
\setlength{\itemsep}{0.05in}
\item{More on Python functions}
\item{Manipulating arrays and strings}
\item{Lists and tuples}
\item{Dictionaries}
\end{enumerate}

\end{frame}


\end{document}

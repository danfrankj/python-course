\documentclass{article}
\newcommand{\assgnnum}{1}
\newcommand{\assigndate}{January 10}

\usepackage{amsmath}
%\usepackage{fullpage}
\usepackage{amssymb}
%\usepackage{bbm}
\usepackage{fancyhdr}
%\usepackage{paralist}
\usepackage{graphicx}
\usepackage[pdftex,colorlinks=true, urlcolor = blue]{hyperref}
\usepackage{arbenson-math}

\oddsidemargin 0in \evensidemargin 0in
\topmargin -0.5in \headheight 0.25in \headsep 0.25in
\textwidth 6.5in \textheight 9in
\parskip 6pt \parindent 0in \footskip 20pt

% set the header up
\fancyhead{}
\fancyhead[L]{CME193: In-class exercises \assgnnum}
\fancyhead[R]{\assigndate}
%%%%%%%%%%%%%%%%%%%%%%%%%%
\renewcommand\headrulewidth{0.4pt}
\setlength\headheight{15pt}


\newcommand{\p}{\ensuremath{\mathbf{P}}}
\renewcommand{\Pr}[1]{\ensuremath{\p \left \{ #1 \right \}}}
\newcommand{\nti}{\ensuremath{n \to \infty}}
\newcommand{\I}{\ensuremath{\operatorname{I}}}
\newcommand{\One}[1]{\ensuremath{\mathbbm{1}_{\left \{ #1 \right \}}}}
\newcommand{\E}{\ensuremath{\mathbf{E}}}
\newcommand{\Ex}[2][]{\ensuremath{\E_{#1} \left[ #2 \right]}}
\newcommand{\var}{\ensuremath{\operatorname{Var}}}
\newcommand{\cov}{\ensuremath{\operatorname{Cov}}}
\newcommand{\F}{\ensuremath{\mathcal{F}}}
\newcommand{\R}{\ensuremath{\mathbb{R}}}
\newcommand{\C}{\ensuremath{\mathbb{C}}}
\newcommand{\NormRV}[2]{\ensuremath{\operatorname{N}\left(#1, #2\right)}}
\newcommand{\BetaRV}[2]{\ensuremath{\operatorname{Beta}\left(#1, #2\right)}}
\newcommand{\argmax}{\operatornamewithlimits{argmax}}
\newcommand{\x}{\mathbf{x}}
\newcommand{\A}{\mathbf{A}}
\newcommand{\bb}{\mathbf{b}}


\newcounter{points}
\setcounter{points}{0}

\newcommand\setpoints[1]{\addtocounter{points}{#1}(#1 points)}
\newcommand\printpoints{Total number of points: \thepoints}

\newcommand{\eqD}{\ensuremath{\overset{\mathcal{D}}{=}}}

\setlength{\parindent}{0in}

\begin{document}

\pagestyle{fancy}
%\vspace*{15pt}
\begin{enumerate}
\item \textbf{Arithmetic and Logic} \\
In class, we saw the arithmetic operators \texttt{+}, \texttt{-}, \texttt{*}, and \texttt{/} and the boolean operators \texttt{and}, \texttt{or}, and \texttt{not}.  In this exercise, we explore more about arithmetic and boolean operators.  Use the Python interpreter to answer the following questions.

\begin{enumerate}
\item What is the difference between the expressions \texttt{x = 7 / 2}, \texttt{x = 7.0 / 2}, and \texttt{x = 7.0 // 2}?  What is the result of \texttt{type(x)} after each expressions? Describe a potential hazard of this arithmetic.  \emph{Note: In Matlab, there is no difference between the first and second expressions.}
\end{enumerate}

\begin{enumerate}
\setcounter{enumii}{1}
\item Let's examine a few other useful operators.  What do the following operators do: \texttt{**}, \texttt{\%}.  Can these operators be used on strings?
\end{enumerate}

\begin{enumerate}
\setcounter{enumii}{2}
\item The operators \texttt{and}, \texttt{or}, and \texttt{not}, corresponding to the more conventional symbols \texttt{\&\&}, \texttt{||}, and \texttt{!} (in Matlab \texttt{\~} instead of \texttt{!}).  To what symbols do the Python operators \texttt{is} and \texttt{is not} correspond? Why is using English words instead of symbols useful?
\end{enumerate}

For more information see:
\begin{itemize}
\item Python expressions: \url{http://docs.python.org/2/reference/expressions.html}
\item Python floating point: \url{http://docs.python.org/2/tutorial/floatingpoint.html}
\end{itemize}



\item \textbf{Control Flow} \\

\begin{enumerate}
\item 
\end{enumerate}



\item \textbf{Functions} \\
Consider the following snippet of Python code: \\

\lstinputlisting{code/deriv1.py}

\begin{enumerate}
\item What gets printed?
\end{enumerate}

\begin{enumerate}
\setcounter{enumii}{1}
\item What is this function doing?
\end{enumerate}

\begin{enumerate}
\setcounter{enumii}{2}
\item Describe some abstractions for this function.  What can be provided as parameters?
\end{enumerate}

Using concepts from the next lecture, here is a much more powerful function:

\begin{center}
\begin{tabular}{c}
\lstinputlisting{code/deriv2.py}
\end{tabular}
\end{center}



\end{enumerate}
%\printpoints.
\end{document} 

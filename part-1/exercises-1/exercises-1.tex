\documentclass{article}
\newcommand{\assgnnum}{1}
\newcommand{\assigndate}{January 10}

\usepackage{amsmath}
%\usepackage{fullpage}
\usepackage{amssymb}
%\usepackage{bbm}
\usepackage{fancyhdr}
%\usepackage{paralist}
\usepackage{graphicx}
\usepackage[pdftex,colorlinks=true, urlcolor = blue]{hyperref}
\usepackage{../../arbenson-math}

\oddsidemargin 0in \evensidemargin 0in
\topmargin -0.5in \headheight 0.25in \headsep 0.25in
\textwidth 6.5in \textheight 9in
\parskip 6pt \parindent 0in \footskip 20pt

% set the header up
\fancyhead{}
\fancyhead[L]{CME193: In-class exercises \assgnnum}
\fancyhead[R]{\assigndate}
%%%%%%%%%%%%%%%%%%%%%%%%%%
\renewcommand\headrulewidth{0.4pt}
\setlength\headheight{15pt}


\newcommand{\p}{\ensuremath{\mathbf{P}}}
\renewcommand{\Pr}[1]{\ensuremath{\p \left \{ #1 \right \}}}
\newcommand{\nti}{\ensuremath{n \to \infty}}
\newcommand{\I}{\ensuremath{\operatorname{I}}}
\newcommand{\One}[1]{\ensuremath{\mathbbm{1}_{\left \{ #1 \right \}}}}
\newcommand{\E}{\ensuremath{\mathbf{E}}}
\newcommand{\Ex}[2][]{\ensuremath{\E_{#1} \left[ #2 \right]}}
\newcommand{\var}{\ensuremath{\operatorname{Var}}}
\newcommand{\cov}{\ensuremath{\operatorname{Cov}}}
\newcommand{\F}{\ensuremath{\mathcal{F}}}
\newcommand{\R}{\ensuremath{\mathbb{R}}}
\newcommand{\C}{\ensuremath{\mathbb{C}}}
\newcommand{\NormRV}[2]{\ensuremath{\operatorname{N}\left(#1, #2\right)}}
\newcommand{\BetaRV}[2]{\ensuremath{\operatorname{Beta}\left(#1, #2\right)}}
\newcommand{\argmax}{\operatornamewithlimits{argmax}}
\newcommand{\x}{\mathbf{x}}
\newcommand{\A}{\mathbf{A}}
\newcommand{\bb}{\mathbf{b}}


\newcounter{points}
\setcounter{points}{0}

\newcommand\setpoints[1]{\addtocounter{points}{#1}(#1 points)}
\newcommand\printpoints{Total number of points: \thepoints}

\newcommand{\eqD}{\ensuremath{\overset{\mathcal{D}}{=}}}

\setlength{\parindent}{0in}

\begin{document}

\pagestyle{fancy}
%\vspace*{15pt}
\begin{enumerate}

\item \textbf{True/False} \\
State whether the following statements are \texttt{True} or \texttt{False} as it would be evaluated in Python (i.e., how it was described in lecture).  Assume that the variable \texttt{x} has a boolean value of \texttt{False} and that the variable \texttt{y} has the value $10$.

\begin{enumerate}
\item
\begin{lstlisting}
x and 8 < y < 12
\end{lstlisting}
\end{enumerate}

\begin{enumerate}
\setcounter{enumii}{1}
\item 
\begin{lstlisting}
'CME ' + '193' == 'cme193'
\end{lstlisting}
\end{enumerate}

\begin{enumerate}
\setcounter{enumii}{2}
\item 
\begin{lstlisting}
y != 12 - 2 or x
\end{lstlisting}
\end{enumerate}

\begin{enumerate}
\setcounter{enumii}{3}
\item 
\begin{lstlisting}
'py' * 2 + 'thonic' == 'pypythonic'
\end{lstlisting}
\end{enumerate}

\item \textbf{Arithmetic} \\
State what $x$ is after each of the following scripts is executed.

\begin{enumerate}
\item \lstinputlisting{code/arith1.py}
\end{enumerate}

\begin{enumerate}
\setcounter{enumii}{1}
\item \lstinputlisting{code/arith2.py}
\end{enumerate}

\begin{enumerate}
\setcounter{enumii}{2}
\item \lstinputlisting{code/arith3.py}
\end{enumerate}

\begin{enumerate}
\setcounter{enumii}{3}
\item \lstinputlisting{code/arith4.py}
\end{enumerate}

\begin{enumerate}
\setcounter{enumii}{4}
\item \lstinputlisting{code/arith5.py}
\end{enumerate}


\item \textbf{Functions and Flow} \\
For each of the following Python scripts, state what gets printed.

\begin{enumerate}
\item \lstinputlisting{code/fnf_a.py}
\end{enumerate}

\begin{enumerate}
\setcounter{enumii}{1}
\item \lstinputlisting{code/fnf_b.py}
\end{enumerate}

\begin{enumerate}
\setcounter{enumii}{2}
\item \lstinputlisting{code/fnf_c.py}
\end{enumerate}

\begin{enumerate}
\setcounter{enumii}{3}
\item The \texttt{elif} statement combines the concepts of an \texttt{else} and an \texttt{if} statement.  It follows an \texttt{if} statement.  If the \texttt{if} statement is false, then the \texttt{elif} statement is evaluated.  If the \texttt{elif} statement is true, that code block executes.

\lstinputlisting{code/fnf_d.py}
\end{enumerate}


\item \textbf{Applications} \\
Consider the following snippet of Python code: \\

\lstinputlisting{code/deriv1.py}

\begin{enumerate}
\item What gets printed?
\end{enumerate}

\begin{enumerate}
\setcounter{enumii}{1}
\item What is this function doing?
\end{enumerate}

\begin{enumerate}
\setcounter{enumii}{2}
\item Describe some abstractions for this function.  What can be provided as parameters?
\end{enumerate}

Using concepts from the next lecture, here is a much more powerful function:

\begin{center}
\begin{tabular}{c}
\lstinputlisting{code/deriv2.py}
\end{tabular}
\end{center}



\end{enumerate}
%\printpoints.
\end{document} 

\documentclass{article}
\newcommand{\assgnnum}{2}
\newcommand{\assigndate}{April 9}

\usepackage{amsmath}
\usepackage{fullpage}
\usepackage{amssymb}
%\usepackage{bbm}
\usepackage{fancyhdr}
%\usepackage{paralist}
\usepackage{graphicx}
\usepackage[pdftex,colorlinks=true, urlcolor = blue]{hyperref}
\usepackage{../../arbenson-math}

\oddsidemargin 0in \evensidemargin 0in
\topmargin -0.5in \headheight 0.25in \headsep 0.25in
\textwidth 6.5in \textheight 9in
\parskip 6pt \parindent 0in \footskip 20pt

% set the header up
\fancyhead{}
\fancyhead[L]{CME193: In-class exercises \assgnnum}

%%%%%%%%%%%%%%%%%%%%%%%%%%
\renewcommand\headrulewidth{0.4pt}
\setlength\headheight{15pt}


\newcommand{\p}{\ensuremath{\mathbf{P}}}
\renewcommand{\Pr}[1]{\ensuremath{\p \left \{ #1 \right \}}}
\newcommand{\nti}{\ensuremath{n \to \infty}}
\newcommand{\I}{\ensuremath{\operatorname{I}}}
\newcommand{\One}[1]{\ensuremath{\mathbbm{1}_{\left \{ #1 \right \}}}}
\newcommand{\E}{\ensuremath{\mathbf{E}}}
\newcommand{\Ex}[2][]{\ensuremath{\E_{#1} \left[ #2 \right]}}
\newcommand{\var}{\ensuremath{\operatorname{Var}}}
\newcommand{\cov}{\ensuremath{\operatorname{Cov}}}
\newcommand{\F}{\ensuremath{\mathcal{F}}}
\newcommand{\R}{\ensuremath{\mathbb{R}}}
\newcommand{\C}{\ensuremath{\mathbb{C}}}
\newcommand{\NormRV}[2]{\ensuremath{\operatorname{N}\left(#1, #2\right)}}
\newcommand{\BetaRV}[2]{\ensuremath{\operatorname{Beta}\left(#1, #2\right)}}
\newcommand{\argmax}{\operatornamewithlimits{argmax}}
\newcommand{\x}{\mathbf{x}}
\newcommand{\A}{\mathbf{A}}
\newcommand{\bb}{\mathbf{b}}


\newcounter{points}
\setcounter{points}{0}

\newcommand\setpoints[1]{\addtocounter{points}{#1}(#1 points)}
\newcommand\printpoints{Total number of points: \thepoints}

\newcommand{\eqD}{\ensuremath{\overset{\mathcal{D}}{=}}}

\setlength{\parindent}{0in}

\begin{document}

\pagestyle{fancy}
%\vspace*{15pt}
\begin{enumerate}

\item \textbf{QR factorization}
\begin{enumerate}
\item Create a function \texttt{qr2} that takes as input two matrices $A$ and $B$ and outputs the $QR$ factorization of the matrix 

\[
\begin{pmatrix} A \\ B\end{pmatrix}
\]

by using the \texttt{hstack} function.  What is the output of the following code snippet?

\vspace{0.5cm}
\lstinputlisting{code/qr1.py}

\end{enumerate}

\begin{enumerate}
\setcounter{enumii}{1}
\item Write a function \texttt{qr3} that takes as input two matrices $A$ and $B$ and does the following:

\begin{enumerate}
\item Computes the QR factorization of $A$: $A = Q_AR_A$.
\item Computes the QR factorization of $B$: $A = Q_BR_B$.
\item Computes the QR factorization of $C = \begin{pmatrix}R_A \\ R_B \end{pmatrix}$: $C = Q_CR_C$.
\item Outputs the tuple $(Q_A, Q_B, Q_C, R_C)$
\end{enumerate}

\vspace{0.5cm}

What is the output of the following code snippet?

\vspace{0.5cm}
\lstinputlisting{code/qr2.py}

How does this compare to the output in part (a)?

\end{enumerate}

\begin{enumerate}
\setcounter{enumii}{2}
\item Using the outputs from part (b), compute the matrix

\[
Q_D = \begin{pmatrix} Q_A & 0_{m_1 \times n} \\ 0_{m_2 \times n} & Q_B \end{pmatrix}Q_C
\]

$0_{m_1 \times n}$ means a matrix of all zeros consisting of $m_1$ rows and $n$ columns, where $m_1$ is the number of rows of $Q_A$ and $n$ is the number of columns of $Q_B$. \\

How does $Q_D$ compare to the output $Q$ from part (a)?

\end{enumerate}

\newpage
\item \textbf{Low-rank approximations with the SVD} \\

\begin{enumerate}
\item Write a function \texttt{laplace} that takes as input a vector $v$ of length $n$ and outputs a matrix $K$ of size $n \times n$ such that
\[
K_{ij} = \left\{
     \begin{array}{lr}
       \frac{1}{|v_i - v_j|}, \quad i \neq j \\
       0, \quad i = j
     \end{array}
   \right.
\]
\end{enumerate}

When the elements of $v$ are points in three-dimensional space, $1 / \| v_i - v_j \|_2$ is called the Laplace kernel.

\begin{enumerate}
\setcounter{enumii}{1}
\item Define $v$ by \texttt{v = np.arange(0.1, 10, 0.5)}.  $v$ should have length $20$.  Define $K$ by \texttt{K = laplace(v)} and let
\[
K_1 = K[0:10, 10:20],
\]
i.e., $K_1$ is the $10 \times 10$ upper-right block of $K$.  Plot the singular values of $K_1$.  What do you notice?
\end{enumerate}


\begin{enumerate}
\setcounter{enumii}{2}
\item Consider the matrix $K_1$ from part (b) and its singular value decomposition
\[
K_1 = U\Sigma V^T.
\]
Compute the matrix $K_2$, defined by
\[
K_2 = U[:, 0:2]\Sigma[0:2, 0:2] V^T[0:2, :].
\]
In other words, $K_2$ is the product of the first three columns of $U$, the upper-left $3 \times 3$ block of $\Sigma$, and the first three rows of $V^T$. $K_2$ is called a \emph{low-rank approximation} of $K_1$. \\

Define the vector $x$ by \texttt{x = np.ones(10)}.  Compute the vector
\[
K_1x - K_2x
\]

Explain the output.

\end{enumerate}




\end{enumerate}
%\printpoints.
\end{document} 

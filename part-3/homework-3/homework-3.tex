\documentclass{article}
\newcommand{\assgnnum}{3}
\newcommand{\duedate}{January 22}

\usepackage{amsmath}
\usepackage{fullpage}
\usepackage{amssymb}
%\usepackage{bbm}
\usepackage{fancyhdr}
%\usepackage{paralist}
\usepackage{graphicx}
\usepackage{caption}
\usepackage{subcaption}
\usepackage[pdftex,colorlinks=true, urlcolor = blue]{hyperref}
\usepackage{listings}
\usepackage{color}
\usepackage{xcolor}

\definecolor{thegreen}{rgb}{0,.5,0}
\definecolor{comment-green}{rgb}{0,.3,0}
\definecolor{theblue}{rgb}{0,0,.8}
\definecolor{light-gray}{gray}{0.98}
\definecolor{comment-color}{rgb}{0,0,.8}
\definecolor{string-color}{rgb}{0,.75,0}
\definecolor{border-blue}{rgb}{0,0,.6}

\lstset{% use our version of highlighting                                            
  language=python,      % using python                                               
  keywordstyle={\color{teal}\bfseries},          % keywords                          
  commentstyle=\color{comment-color},         % comments                             
  stringstyle=\color{string-color},                   %strings                       
}

\lstset{
  basicstyle={\ttfamily\normalsize},  % use font and smaller size                    
  basewidth={0.5em,0.5em},
  showstringspaces=false,                   % do not emphasize spaces in strings     
  tabsize=2,                                % number of spaces of a TAB              
  aboveskip={0\baselineskip},               % a bit of space above                   
  columns=fixed,                            % nice spacing                           
}


\oddsidemargin 0in \evensidemargin 0in
\topmargin -0.5in \headheight 0.25in \headsep 0.25in
\textwidth 6.5in \textheight 9in
\parskip 6pt \parindent 0in \footskip 20pt

% set the header up
\fancyhead{}
\fancyhead[L]{CME193: Assignment \assgnnum}
\fancyhead[R]{Due: \duedate}
%%%%%%%%%%%%%%%%%%%%%%%%%%
\renewcommand\headrulewidth{0.4pt}
\setlength\headheight{15pt}

\newcommand{\p}{\ensuremath{\mathbf{P}}}
\renewcommand{\Pr}[1]{\ensuremath{\p \left \{ #1 \right \}}}
\newcommand{\nti}{\ensuremath{n \to \infty}}
\newcommand{\I}{\ensuremath{\operatorname{I}}}
\newcommand{\One}[1]{\ensuremath{\mathbbm{1}_{\left \{ #1 \right \}}}}
\newcommand{\E}{\ensuremath{\mathbf{E}}}
\newcommand{\Ex}[2][]{\ensuremath{\E_{#1} \left[ #2 \right]}}
\newcommand{\var}{\ensuremath{\operatorname{Var}}}
\newcommand{\cov}{\ensuremath{\operatorname{Cov}}}
\newcommand{\F}{\ensuremath{\mathcal{F}}}
\newcommand{\R}{\ensuremath{\mathbb{R}}}
\newcommand{\C}{\ensuremath{\mathbb{C}}}
\newcommand{\NormRV}[2]{\ensuremath{\operatorname{N}\left(#1, #2\right)}}
\newcommand{\BetaRV}[2]{\ensuremath{\operatorname{Beta}\left(#1, #2\right)}}
\newcommand{\argmax}{\operatornamewithlimits{argmax}}
\newcommand{\x}{\mathbf{x}}
\newcommand{\A}{\mathbf{A}}
\newcommand{\bb}{\mathbf{b}}

\newcommand{\ssum}[3]{\displaystyle\sum\limits_{#1}^{#2} #3}

\newcounter{points}
\setcounter{points}{0}
\newcounter{bonuspoints}
\setcounter{bonuspoints}{0}

\newcommand\setpoints[1]{\addtocounter{points}{#1}(#1 points)}
\newcommand\setpoint{\addtocounter{points}{1}(1 point)}
\newcommand\setbonuspoints[1]{\addtocounter{bonuspoints}{#1}(#1 bonus points)}
\newcommand\setbonuspoint{\addtocounter{bonuspoints}{1}(1 bonus point)}

\newcommand\printpoints{Total number of points: \value{\thepoints}}

\newcommand{\eqD}{\ensuremath{\overset{\mathcal{D}}{=}}}

\setlength{\parindent}{0in}

\begin{document}

\pagestyle{fancy}
%\vspace*{15pt}

50 points total.  70+\% correctness (35+ points) is needed to pass.  There is also the opportunity for 5 bonus points.  Remember: you must pass all assignments to pass the class.  The assignment is due at the beginning of the next class.

\begin{enumerate}
\item \textbf{Textual analysis} \\
An \emph{n-gram} is a sequence of $n$ consecutive words in a text.  For example, the $2$-grams of:
\begin{center}
 ``I love the Python programming language"
\end{center} are ``I love", ``love the", ``the Python", ``Python programming", and ``programming langue";  the $3-$grams are ``I love the", "love the Python", "the Python programming", and "Python programming language"; and so on for larger values of $n$.  We say that there are no $7$-grams for this sentence because there are only $6$ words.

$n$-grams are sometimes used to analyze patterns in text (for example, see Google's Ngram Viewer at \url{http://books.google.com/ngrams}).  In this assignment, you are going to implement a function that computes the $k$ most frequently occurring $n$-grams in a text file. \\

Implement \texttt{ngram()} in the file \texttt{ngram.py}.  The function definition is: \\

\begin{lstlisting}
                       def ngram(n, k, document):
\end{lstlisting}
$n$ specifies that we are computing $n$-grams, $k$ specifies that we want the $k$ most frequently occurring $n$-grams (breaking ties arbitrarily), and \emph{document} is the name of a file containing the text.  If there are fewer than $k$ $n$-grams, return all $n$-grams).  The return value should be a dictionary with keys given by $n$-grams and values given by $n$-gram frequency in the text  (if no $n$-grams are found, say for $n$ too big, return an empty dictionary).  \\

Record the output from \texttt{test1()} and \texttt{test2()}.  These tests use the files \texttt{course\_description.txt} and \texttt{bill\_of\_rights.txt}, which were packaged with the starter code. \setpoints{20} \\

Assume that the file \texttt{document} contains no punctuation.  Also assume that all words are separated by a single space. Note that capital letters constitute different words, so ``Python programming" and``python programming" would be counted as different 2-grams. \\

Finally, only compute $n$-grams that occur on a single line of the text (not $n$-grams that contain words from the end of one line and the beginning of the next line).  Therefore, you can follow the examples from lecture on reading a file line-by-line. \\

\item \textbf{Money classes} \\
The file \texttt{stocks.py} contains the skeleton code for a stock portfolio class.  This class is similar in spirit to the one presented in class, but it is a different class with different functionality.

\begin{enumerate}
\item Implement the \texttt{add\_stock()} and \texttt{most\_money()} functions.  What gets printed from calling the function \texttt{test1()}? \setpoints{8}
\end{enumerate}

\begin{enumerate}
\setcounter{enumii}{1}
\item Implement the \texttt{\_\_contains\_\_()} function.  Remember from lecture that this is a special function that works with the \texttt{in} operator.  What gets printed from calling the function \texttt{test2()}? \setpoints{7}
\end{enumerate}

\newpage
\item \textbf{NumPy} \\
In this problem, we will explore the basics of NumPy.  Place all of your code into a file called \texttt{myfirstnumpy.py}.

\begin{enumerate}
\item Download and install NumPy on your machine.  Record the output of the following lines.  \setpoints{3} \\
\texttt{>>> import numpy as np} \\
\texttt{>>> np.eye(3, 3)} \\
\end{enumerate}

\begin{enumerate}
\setcounter{enumii}{1}
\item The files \texttt{a.txt} and \texttt{b.txt} packaged with the starter code each contain data for a vector.  Each row of each file contains one floating point number.  Read the files \texttt{a.txt} and \texttt{b.txt} and store the data into two lists, \texttt{a} and \texttt{b}.  What are the lengths of these lists?  They should be the same.  \setpoints{3}
\end{enumerate}

\begin{enumerate}
\setcounter{enumii}{2}
\item Look at the online documentation for NumPy on the dot product function.  What is the name of the dot product function?  Compute the dot product of $a$ and $b$ using this function. \setpoints{3}
\end{enumerate}

\begin{enumerate}
\setcounter{enumii}{3}
\item Convert your vectors to the \texttt{numpy.array} data type.  Use the normal multiplication operator \texttt{*} to compute the entry-wise multiplication of the vectors into a new vector, \texttt{c}.  Write the elements of \texttt{c} to the file \texttt{c.txt}, one entry per line (the same as the representations in \texttt{a} and \texttt{b}). \setpoints{3} \\

Remember the dot product of two vectors of length $n$, $x$ and $y$, is given by $\ssum{i=1}{n}{x_iy_i}$.
\end{enumerate}

\begin{enumerate}
\setcounter{enumii}{4}
\item The one norm of a vector $x$ of length $n$ is given by:
\[
||x||_1 = \ssum{i=1}{n}{|x_i|},
\]
 i.e., the sum of the absolute values of the entries. \\

Compute the one norm of the vector $a - b$.  With the \texttt{numpy.array} data type, you can use the normal subtraction operator \texttt{-} to compute $a - b$.  Do not manually compute the norm.  Look up the NumPy norm function (what is it?) and use it to compute the norm. \setpoints{3}.
\end{enumerate}

\item \textbf{Python libraries (bonus)} \\
There are a vast number of application-specific Python libraries.  For this problem, find a Python library related to your major/graduate program/interests and briefly explain what it does.  If you choose NumPy/SciPy, choose a specific application area or function.  Create a sample snippet of code that uses the library.  Place the code in a file called \texttt{bonus.py}.  \setbonuspoints {5} 


\item \textbf{Submission} \\
Upload the following files to your Drop Box on Coursework:
\begin{enumerate}
\item \texttt{ngram.py}
\item \texttt{stocks.py}
\item \texttt{myfirstnumpy.py}
\item \texttt{c.txt}
\item \texttt{bonus.py} (if applicable)
\end{enumerate}

For the questions that required non-code answers, either type up your solutions and put them in your Drop Box or bring a hard copy to the next lecture.  If you upload to Drop Box, please upload a PDF file.

\end{enumerate}
%\printpoints.
\end{document} 

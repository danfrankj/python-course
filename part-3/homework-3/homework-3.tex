\documentclass{article}
\newcommand{\assgnnum}{3}
\newcommand{\duedate}{January 22}

\usepackage{amsmath}
\usepackage{fullpage}
\usepackage{amssymb}
%\usepackage{bbm}
\usepackage{fancyhdr}
%\usepackage{paralist}
\usepackage{graphicx}
\usepackage{caption}
\usepackage{subcaption}
\usepackage[pdftex,colorlinks=true, urlcolor = blue]{hyperref}
\usepackage{../../arbenson-math}


\oddsidemargin 0in \evensidemargin 0in
\topmargin -0.5in \headheight 0.25in \headsep 0.25in
\textwidth 6.5in \textheight 9in
\parskip 6pt \parindent 0in \footskip 20pt

% set the header up
\fancyhead{}
\fancyhead[L]{CME193: Assignment \assgnnum}
\fancyhead[R]{Due: \duedate}
%%%%%%%%%%%%%%%%%%%%%%%%%%
\renewcommand\headrulewidth{0.4pt}
\setlength\headheight{15pt}

\newcommand{\p}{\ensuremath{\mathbf{P}}}
\renewcommand{\Pr}[1]{\ensuremath{\p \left \{ #1 \right \}}}
\newcommand{\nti}{\ensuremath{n \to \infty}}
\newcommand{\I}{\ensuremath{\operatorname{I}}}
\newcommand{\One}[1]{\ensuremath{\mathbbm{1}_{\left \{ #1 \right \}}}}
\newcommand{\E}{\ensuremath{\mathbf{E}}}
\newcommand{\Ex}[2][]{\ensuremath{\E_{#1} \left[ #2 \right]}}
\newcommand{\var}{\ensuremath{\operatorname{Var}}}
\newcommand{\cov}{\ensuremath{\operatorname{Cov}}}
\newcommand{\F}{\ensuremath{\mathcal{F}}}
\newcommand{\R}{\ensuremath{\mathbb{R}}}
\newcommand{\C}{\ensuremath{\mathbb{C}}}
\newcommand{\NormRV}[2]{\ensuremath{\operatorname{N}\left(#1, #2\right)}}
\newcommand{\BetaRV}[2]{\ensuremath{\operatorname{Beta}\left(#1, #2\right)}}
\newcommand{\argmax}{\operatornamewithlimits{argmax}}
\newcommand{\x}{\mathbf{x}}
\newcommand{\A}{\mathbf{A}}
\newcommand{\bb}{\mathbf{b}}

\newcounter{points}
\setcounter{points}{0}

\newcommand\setpoints[1]{\addtocounter{points}{#1}(#1 points)}
\newcommand\printpoints{Total number of points: \thepoints}

\newcommand{\eqD}{\ensuremath{\overset{\mathcal{D}}{=}}}

\setlength{\parindent}{0in}

\begin{document}

\pagestyle{fancy}
%\vspace*{15pt}

40 points total.  70+\% correctness (28+ points) is needed to pass.  There is also the opportunity for 5 bonus points.  Remember: you must pass all assignments to pass the class.  The assignment is due at the beginning of the next class.  For the grading details, see question 2, part (e).

\begin{enumerate}
\item \textbf{Textual analysis} \\
An \emph{n-gram} is a sequence of $n$ consecutive words in a text.  For example, the $2$-grams of:
\begin{center}
 ``I love the Python programming language"
\end{center} are ``I love", ``love the", ``the Python", ``Python programming", and ``programming langue";  the $3-$grams are ``I love the", "love the Python", "the Python programming", and "Python programming language"; and so on for larger values of $n$.  We say that there are no $7$-grams for this sentence because there are only $6$ words.

$n$-grams are sometimes used to analyze patterns in text (for example, see Google's Ngram Viewer at \url{http://books.google.com/ngrams}).  In this assignment, you are going to implement a function that computes the $k$ most frequently occurring $n$-grams in a text file.

\begin{enumerate}
\item Implement \texttt{ngram()} in the file \texttt{ngram.py}.  The function definition is: \\

\begin{lstlisting}
                       def ngram(n, k, text):
\end{lstlisting}
$n$ specifies that we are computing $n$-grams, $k$ specifies that we want the $k$ most frequently occurring $n$-grams, and text is the name of a file containing the text.  If there are fewer than $k$ $n$-grams, return all $n$-grams.  The return value should be a dictionary with keys given by $n$-grams and values given by $n$-gram frequency in the text. \\

Assume that the file \texttt{text} contains no punctuation.  Also assume that all words are separated by a single space. Note that capital letters constitute different words, so ``Python programming" and``python programming" would be counted as different 2-grams. \\

Finally, only compute $n$-grams that occur on a single line of the text (not $n$-grams that contain words from the end of one line and the beginning of the next line).  Therefore, you can follow the examples from lecture on reading a file line-by-line.
\end{enumerate}

\begin{enumerate}
\setcounter{enumii}{1}
\item Similar to last week, an autograder has been provided.
\end{enumerate}

\item \textbf{Classes} \\
In the last homework, we 

\item \textbf{Python libraries} \\
One reason to love Python is the vast amount of libraries available.  For this problem, find a Python library related to your major/graduate program/interests and briefly explain what it does.  If you choose NumPy/SciPy, choose a specific application area or function.  Give a sample snippet of code that uses the library.  An example is provided below. \\






\item \textbf{Submission} \\



\end{enumerate}
%\printpoints.
\end{document} 

\documentclass{beamer}
\usepackage{caption}
\usepackage{subcaption}
\usepackage{../../arbenson-math}

\usetheme{boxes}
\usecolortheme{seahorse}

\AtBeginSection[]
{
  \begin{frame}<beamer>
    \frametitle{\thesection}
    \tableofcontents[currentsection]
  \end{frame}
}

\title{CME 193: Introduction to Scientific Python \\
Lecture 3: File I/O, Object-oriented Python, and Intro to NumPy}
\author{Austin Benson \\
\vspace{0.1in}
Dan Frank \\
\vspace{0.1in}
Institute for Computational and Mathematical Engineering (ICME)}
\date{January 17, 2013}
\begin{document}
\maketitle

\section{Administrivia}
\begin{frame}
\frametitle{Homework}
\begin{itemize}
\setlength{\itemsep}{0.1in}
\item{Overall, good job on homework 1.}
\item{Homework 2 is due right now.}
\item{Homework 3 is posted (longer than the first two assignments).}
\end{itemize}
\end{frame}

\begin{frame}
\frametitle{Homework 1: is and is not}
\texttt{==} determines if the objects have the same value.  \texttt{is} checks if the objects are the same.
\codeblock{code/is_comp.py}
\end{frame}

\begin{frame}
\frametitle{Homework 1: \% operator on strings}
There is no analogous modulo operator on strings.  However, \texttt{\%} is used for string formatting:
\codeblock{code/string_formatting.py}
\end{frame}

\begin{frame}
\frametitle{Homework 1: style}
\codeblock{code/return_style1.py}
\end{frame}

\begin{frame}
\frametitle{Homework 1: style}
\codeblock{code/return_style2.py}
\end{frame}

\section{File I/O}
\begin{frame}
\frametitle{Programming Languages and the system}
\begin{itemize}
\setlength{\itemsep}{0.1in}
\item{Matlab encourages a separation between the program and the computer system}
\item{C provides powerful ways to control the computer system, but the code is verbose}
\item{Python makes it very easy to interact with the computer system in basic ways}
\end{itemize}
\end{frame}

\begin{frame}
\frametitle{Python and the system}
\begin{itemize}
\setlength{\itemsep}{0.1in}
\item{For scientific computing, this balance between power and ease of programming makes Python a popular choice}
\item{Today, we will focus on file I/O as our interaction with the computer system}
\item{Please check out the \texttt{os} library (operating system): \url{http://docs.python.org/2/library/os.html}.}
\end{itemize}
\end{frame}

\begin{frame}
\frametitle{File reading}
Suppose we have a text file of chemical compounds:
\codeblock{code/compounds.txt}
We want to read this file and store the information in a dictionary.
\end{frame}

\begin{frame}
\frametitle{File reading}
\codeblock{code/compounds1.py}
'r'  specifies that we want to read the file. \\
\vspace{0.1in}
\emph{f} is now a file object
\end{frame}

\begin{frame}
\frametitle{File reading}
Print the entire contents of the file:
\codeblock{code/compounds2.py}
\end{frame}

\begin{frame}
\frametitle{File reading}
Print each line individually:
\codeblock{code/compounds3.py}

\vspace{0.3in}
Prints ``(Line \#1) salt: NaCl", etc.
\end{frame}


\begin{frame}
\frametitle{File reading}
A verbose dictionary formulation:
\codeblock{code/compounds4.py}
\end{frame}


\begin{frame}
\frametitle{File reading}
The ``with" statement closes the file automatically.  A more Pythonic implementation:
\codeblock{code/compounds5.py}
\end{frame}

\begin{frame}
\frametitle{File reading}
One liner:
\codeblock{code/compounds6.py}

\vspace{0.3in}
... might be a bit much for one line.
\end{frame}


\begin{frame}
\frametitle{File writing}
Now we are going the other way.  We have data in our Python program that we want to store.
\end{frame}

\begin{frame}
\frametitle{File writing}
Suppose we have scraped some web data from \url{www.reddit.com}.
\end{frame}

\begin{frame}
\frametitle{File writing}
\texttt{reddit\_data.py}:
\codeblock{code/reddit_data.py}
\end{frame}

\begin{frame}
\frametitle{File writing}
\codeblock{code/reddit1.py}
Note that we request write permission with `w' when opening the file.
\end{frame}

\begin{frame}[fragile]
\frametitle{File writing}
Result:
\begin{verbatim}
{'sub': 'aww', 'comments': 595, 'title': 'The eyes says it all'}
{'sub': 'AskReddit', 'comments': 6494, 'title': 'From typical youtube upload to serendipity in 30 seconds'}
{'sub': 'AdviceAnimals', 'comments': 95, 'title': "Use a decent host or don't even try at all..."}
\end{verbatim}
\vspace{0.2in}
any problems with this output?
\end{frame}

\begin{frame}
\frametitle{File writing}
We might want a little more structure to the output file:
\codeblock{code/reddit2.py}
\end{frame}

\begin{frame}[fragile]
\frametitle{File writing}
Result:
\begin{verbatim}
Post #0
        The eyes says it all
                aww (595)
Post #1
        From typical youtube
                AskReddit (6494)
Post #2
        Use a decent host or
                AdviceAnimals (95)
\end{verbatim}
\end{frame}

\section{Python Classes}

\begin{frame}
\frametitle{Classes}

Classes:
\vspace{0.1in}
\begin{itemize}
\setlength{\itemsep}{0.1in}
\item{containers of data, information, and ideas}
\item{basis for object-oriented programming}
\end{itemize}

\end{frame}

\begin{frame}
\frametitle{Python Classes}

More specifically, Python classes:
\vspace{0.1in}
\begin{itemize}
\setlength{\itemsep}{0.1in}
\item{contain ``instance variables" as data}
\item{contain functions (sometimes called ``methods" in the context of classes)}
\item{structure can change on the fly}
\end{itemize}

\end{frame}

\begin{frame}
\frametitle{Differences from traditional OO}

In languages like C++ and Java, classes provide data protection (public/private functions, friend classes, etc.).  In Python, we just get the basics like inheritance.

\vspace{0.2in}

It is up to the programmer to not abuse the classes.  This works well in practice, and the code remains simple.

\end{frame}

\begin{frame}
\frametitle{Stock Prices}
\codeblock{code/stocks1.py}

\end{frame}


\begin{frame}
\frametitle{Constructors}

The \texttt{\_\_init\_\_()} function is the special class constructor.  It is the function that gets called when we make the statement:

\begin{center}
\texttt{Stock('Google', 'GOOG')}.
\end{center}

\end{frame}

\begin{frame}
\frametitle{self}
The self parameter is a little weird.

\vspace{0.2in}

The self variable is a pointer to the class that you are modifying.  For example:

\begin{center}
\texttt{self.symbol = symbol}.
\end{center}

says to modify the variable instance variable \texttt{symbol} in this class instantiation.  On the right-hand-side, symbol is the name of a local variable (from the parameters).

\end{frame}

\begin{frame}
\frametitle{Functions in classes}
Classes can have functions:

\codeblock{code/stocks2.py}
\end{frame}


\begin{frame}
\frametitle{Functions in classes}
Notice how the \texttt{high\_price()} function uses \texttt{self} to get the maximum price from that particular stock.
\end{frame}


\begin{frame}
\frametitle{Glorified dictionaries?}
If you think that classes are like dictionaries, you are right:
\codeblock{code/stocks3.py}
\end{frame}

\begin{frame}
\frametitle{Glorified dictionaries?}
The dictionary version is messy, and classes are cleaner.

\vspace{0.2in}

The subject of how to implement classes is material for a Programming Languages/Compilers course.
\end{frame}

\begin{frame}
\frametitle{Inheritance}
Inheritance is a way for classes to share structure.

\vspace{0.2in}
A class can ``inherit" the functions and data from a parent class.
\end{frame}

\begin{frame}
\frametitle{Stock options}
A stock option is like a stock.  When purchasing a stock option, we purchase the ``right to buy" the stock at a certain price at a certain time in the future.

\vspace{0.2in}
We want to augment our Stock class with information about the option.
\end{frame}

\begin{frame}
\frametitle{Stock options}
\codeblock{code/stocks4.py}
\end{frame}

\begin{frame}
\frametitle{Stock options}
The \texttt{high\_price()} method in the \texttt{StockOption} class is inherited from the \texttt{Stock} class.

\vspace{0.2in}
Alternatively, we could override the method.
\end{frame}

\begin{frame}
\frametitle{Override}
\codeblock{code/stocks5.py}
\end{frame}

\begin{frame}
\frametitle{Python goodies}
\codeblock{code/stocks6.py}
\end{frame}

\begin{frame}
\frametitle{Python goodies}
The \text{\_\_contains\_\_()} function is a special class function designed to work with the \texttt{in} operator.

\vspace{0.2in}

There are other special class functions.  For example, there is one for iterators (\texttt{for item in my\_class}).

\end{frame}


\section{Intro to NumPy}

\begin{frame}
\frametitle{NumPy}
\codeblock{code/numpy1.py}
\end{frame}


\begin{frame}
Linear classifier:
\frametitle{NumPy}
\codeblock{code/numpy2.py}
\end{frame}


\begin{frame}
QR factorizations:
\frametitle{NumPy}
\codeblock{code/numpy3.py}
\end{frame}

\begin{frame}
\frametitle{End}

Assignment 3 is posted on the course web site (due Tuesday 1/22).

\vspace{0.2in}

Next time:
\begin{enumerate}
\setlength{\itemsep}{0.05in}
\item{Dan is lecturing}
\item{More NumPy}
\item{SciPy}
\end{enumerate}

\end{frame}

\end{document}

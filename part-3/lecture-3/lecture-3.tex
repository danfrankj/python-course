\documentclass{beamer}
\usepackage{caption}
\usepackage{subcaption}
\usepackage{../../arbenson-math}

\usetheme{boxes}
\usecolortheme{seahorse}

\AtBeginSection[]
{
  \begin{frame}<beamer>
    \frametitle{\thesection}
    \tableofcontents[currentsection]
  \end{frame}
}

\title{CME 193: Introduction to Scientific Python \\
Lecture 3: File I/O and Object-oriented Python}
\author{Austin Benson \\
\vspace{0.1in}
Dan Frank \\
\vspace{0.1in}
Institute for Computational and Mathematical Engineering (ICME)}
\begin{document}
\maketitle

\section{File I/O}
\begin{frame}

\end{frame}

\section{Python Classes}

\begin{frame}
\frametitle{Classes}

Classes:
\vspace{0.1in}
\begin{itemize}
\setlength{\itemsep}{0.1in}
\item{containers of data, information, and ideas}
\item{basis for object-oriented programming}
\end{itemize}

\end{frame}

\begin{frame}
\frametitle{Python Classes}

More specifically, Python classes:
\vspace{0.1in}
\begin{itemize}
\setlength{\itemsep}{0.1in}
\item{contain ``member" data, which are just variables}
\item{contain functions (sometimes called ``methods" in the context of classes)}
\item{structure can change on the fly}
\end{itemize}

\end{frame}

\begin{frame}
\frametitle{Differences from traditional OO}

In languages like C++ and Java, classes provide data protection (public/private functions, friend classes, etc.).  In Python, we just get the basics like inheritance.

\vspace{0.2in}

It is up to the programmer to not abuse the classes.  This works well in practice, and the code remains simple.
\end{frame}

\begin{frame}
\frametitle{Stock Prices}

\codeblock{code/stocks1.py}

\end{frame}


\begin{frame}
\frametitle{Constructors}

The \texttt{\_\_init\_\_()} function is the special class constructor.  It is the function that gets called when we make the statement:

\begin{center}
\texttt{Stock('Google', 'GOOG')}.
\end{center}

\end{frame}

\begin{frame}
\frametitle{self}
The self parameter is a little weird.  We do not actually provide the self value.

\vspace{0.2in}

The self variable is a pointer to the class that you are modifying.  For example:

\begin{center}
\texttt{self.symbol = symbol}.
\end{center}

says to modify the variable \texttt{symbol} in this class.

\end{frame}

\begin{frame}
\frametitle{Functions in classes}
Classes can have functions:

\codeblock{code/stocks2.py}
\end{frame}


\begin{frame}
\frametitle{Functions in classes}
Notice how the \texttt{high\_price()} function uses \texttt{self} to get the maximum price from that particular stock.
\end{frame}


\begin{frame}
\frametitle{Glorified dictionaries?}
If you think that classes are like dictionaries, you are right:
\codeblock{code/stocks3.py}
\end{frame}

\begin{frame}
\frametitle{Glorified dictionaries?}
The dictionary version is messy, and classes are cleaner.

\vspace{0.2in}

The subject of how to implement classes is material for a Programming Languages/Compilers course.
\end{frame}

\begin{frame}
\frametitle{Inheritance}
Inheritance is a way for classes to share structure.

\vspace{0.2in}
A class can ``inherit" the functions and data from a parent class.
\end{frame}

\begin{frame}
\frametitle{Stock options}
A stock option is like a stock.  When purchasing a stock option, we purchase the ``right to buy" the stock at a certain price at a certain time in the future.

\vspace{0.2in}
We want to augment our stock class with information about option.
\end{frame}

\begin{frame}
\frametitle{Stock options}
\codeblock{code/stocks4.py}
\end{frame}

\begin{frame}
\frametitle{Stock options}
The \texttt{high\_price()} method in the \texttt{StockOption} class is inherited from the \texttt{Stock} class.

\vspace{0.2in}
Alternatively, we could override the method.
\end{frame}

\begin{frame}
\frametitle{Override}
\codeblock{code/stocks5.py}
\end{frame}

\begin{frame}
\frametitle{Python goodies}
\codeblock{code/stocks6.py}
\end{frame}

\begin{frame}
\frametitle{Python goodies}
The \text{\_\_contains\_\_()} function is a special class function designed to work with the \texttt{in} operator.

\vspace{0.2in}

There are other special class functions.  For example, there is one for iterators (\texttt{for item in my\_class}).

\end{frame}
%After knowing a few basic types, you can write powerful Python code.

%\vspace{0.1in}

%Data types covered today:
%\begin{itemize}
%\setlength{\itemsep}{0.1in}
%\item{Lists and tuples}
%\item{Strings}
%\item{Dictionaries}
%\end{itemize}

%\end{frame}

%\begin{frame}
%\frametitle{Lists}
%We can also manipulate slices of an array:

%\codeblock{code/lists2.py}

%\end{frame}



\end{document}

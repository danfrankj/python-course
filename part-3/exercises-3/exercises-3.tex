\documentclass{article}
\newcommand{\assgnnum}{3}
\newcommand{\assigndate}{January 22}

\usepackage{amsmath}
%\usepackage{fullpage}
\usepackage{amssymb}
%\usepackage{bbm}
\usepackage{fancyhdr}
%\usepackage{paralist}
\usepackage{graphicx}
\usepackage[pdftex,colorlinks=true, urlcolor = blue]{hyperref}
\usepackage{../../arbenson-math}

\oddsidemargin 0in \evensidemargin 0in
\topmargin -0.5in \headheight 0.25in \headsep 0.25in
\textwidth 6.5in \textheight 9in
\parskip 6pt \parindent 0in \footskip 20pt

% set the header up
\fancyhead{}
\fancyhead[L]{CME193: In-class exercises \assgnnum}
\fancyhead[R]{\assigndate}
%%%%%%%%%%%%%%%%%%%%%%%%%%
\renewcommand\headrulewidth{0.4pt}
\setlength\headheight{15pt}


\newcommand{\p}{\ensuremath{\mathbf{P}}}
\renewcommand{\Pr}[1]{\ensuremath{\p \left \{ #1 \right \}}}
\newcommand{\nti}{\ensuremath{n \to \infty}}
\newcommand{\I}{\ensuremath{\operatorname{I}}}
\newcommand{\One}[1]{\ensuremath{\mathbbm{1}_{\left \{ #1 \right \}}}}
\newcommand{\E}{\ensuremath{\mathbf{E}}}
\newcommand{\Ex}[2][]{\ensuremath{\E_{#1} \left[ #2 \right]}}
\newcommand{\var}{\ensuremath{\operatorname{Var}}}
\newcommand{\cov}{\ensuremath{\operatorname{Cov}}}
\newcommand{\F}{\ensuremath{\mathcal{F}}}
\newcommand{\R}{\ensuremath{\mathbb{R}}}
\newcommand{\C}{\ensuremath{\mathbb{C}}}
\newcommand{\NormRV}[2]{\ensuremath{\operatorname{N}\left(#1, #2\right)}}
\newcommand{\BetaRV}[2]{\ensuremath{\operatorname{Beta}\left(#1, #2\right)}}
\newcommand{\argmax}{\operatornamewithlimits{argmax}}
\newcommand{\x}{\mathbf{x}}
\newcommand{\A}{\mathbf{A}}
\newcommand{\bb}{\mathbf{b}}


\newcounter{points}
\setcounter{points}{0}

\newcommand\setpoints[1]{\addtocounter{points}{#1}(#1 points)}
\newcommand\printpoints{Total number of points: \thepoints}

\newcommand{\eqD}{\ensuremath{\overset{\mathcal{D}}{=}}}

\setlength{\parindent}{0in}

\begin{document}

\pagestyle{fancy}
%\vspace*{15pt}
\begin{enumerate}

\item \textbf{Reading file data}
\begin{enumerate}
\item Write a function that takes as input the name of a text file and returns the most frequently occurring word in the file.  You can test your function with:
\begin{center}
\url{http://stanford.edu/~arbenson/cme193/data/course_description.txt}
\end{center}
\end{enumerate}

\begin{enumerate}
\setcounter{enumii}{1}
\item Write a function that takes as input the name of a text file and returns the total number of words in the file.  You can test your function with:
\begin{center}
\url{http://stanford.edu/~arbenson/cme193/data/bill_of_rights.txt}
\end{center}
\end{enumerate}

\begin{enumerate}
\setcounter{enumii}{2}
\item Suppose that we have scraped some web data from reddit.  We have a text file where each row contains the base web page of the post, the number of comments, and the number of ``upvotes" the post has received.  Suppose that the data for each post is separated by \texttt{|}.  For example, here are 5 data points: \\
\lstinputlisting{code/reddit_data.txt}
Write a function that computes the average number of comments (number of comments is the right-most column) of the data points.  Assume that the name of the data file is the function argument. \\

The above data is available at:
\begin{center}
\url{http://stanford.edu/~arbenson/cme193/data/reddit_data.txt}
\end{center}
\end{enumerate}

\begin{enumerate}
\setcounter{enumii}{3}
\item Write a function that computes the average number of upvotes for posts on imgur and average number of upvotes for posts not on imgur.
\end{enumerate}

\item \textbf{Cartesian points}
\begin{enumerate}
\item In the second homework, you represented $(x, y, z)$ points using tuples and lists.  Write a class called \texttt{CartPoint} that contains x, y, and z points as member variables.  The constructor for the class should take an $(x, y, z)$ tuple.
\end{enumerate}

\begin{enumerate}
\setcounter{enumii}{1}
\item Add a \texttt{magnitude()} function to \texttt{CartPoint} that computes the magnitude of the point ($\sqrt{x^2 + y^2 + z^2}$).
\end{enumerate}

\begin{enumerate}
\setcounter{enumii}{2}
\item Create a subclass of \texttt{CartPoint} called \texttt{CartPointTime} that represents an $(x, y, z, t)$ vector, $t$ representing time.  The constructor for the class should take an $(x, y, z, t)$ tuple.
\end{enumerate}

\begin{enumerate}
\setcounter{enumii}{3}
\item Add a \texttt{magnitude\_and\_time()} to \texttt{CartPointTime} function that returns a $(magnitude, time)$ tuple.
\end{enumerate}

\newpage
\item \textbf{Applications: Gene sequencing} \\
Suppose we are trying to find length-4 patterns in length-10 gene sequences.  We have some Python data structures with information, and we have some methods for managing our data.  However, there is not much organization to these pieces. \\

Create a class called \texttt{GenePatterns} that has similar functionality to the code provided below. \\

\lstinputlisting{code/genes.py}



\end{enumerate}
%\printpoints.
\end{document} 

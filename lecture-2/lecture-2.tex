\documentclass{beamer}
\usetheme{Frankfurt}

\usepackage{caption}
\usepackage{subcaption}
\usepackage{listings}
\usepackage{color}
\usepackage{amsmath}

\definecolor{thegreen}{rgb}{0,.5,0}
\definecolor{theblue}{rgb}{0,0,.8}
\lstset{% use our version of highlighting
  language=python,      % using python
  keywordstyle={\color{theblue}\bfseries},          % keywords
  commentstyle=\color{thegreen},         % comments
}
\lstset{%
  basicstyle={\ttfamily\normalsize},  % use font and smaller size
  basewidth={0.5em,0.5em},
  showstringspaces=false,                   % do not emphasize spaces in strings
  tabsize=2,                                % number of spaces of a TAB
  aboveskip={0\baselineskip},               % a bit of space above
  columns=fixed,                            % nice spacing
}

\AtBeginSection[]
{
  \begin{frame}<beamer>
    \frametitle{\thesection}
    \tableofcontents[currentsection]
  \end{frame}
}

\title{Introduction to Scientific Python \\
Lecture 2: Introduction to Python (Part I)}
\author{Austin Benson \\
\vspace{0.1in}
Dan Frank \\
\vspace{0.1in}
Institue for Computational and Mathematical Engineering (ICME)}
\begin{document}
\maketitle

\begin{frame}
\frametitle{Data types}

After knowing a few basic types, you can write powerful Python code.

\vspace{0.1in}

Data types covered today:
\begin{itemize}
\setlength{\itemsep}{0.1in}
\item{Lists and tuples}
\item{Strings}
\item{Dictionaries}
\end{itemize}

\end{frame}

\section{Lists and tuples}

\begin{frame}
\frametitle{Lists}

Lists (arrays) store a sequence of data which supports indexing.

\vspace{0.1in}

We saw lists in the polynomial evaluation example from last lecture.

\end{frame}

\begin{frame}
\frametitle{Lists}

%\begin{center}
\begin{tabular}{l}
\lstinputlisting{code/lists1.py}
\end{tabular}
%\end{center}

\vspace{0.1in}

Negative indexing!

\end{frame}

\begin{frame}
\frametitle{Lists}
We can also manipulate slices of an array:

\begin{center}
\begin{tabular}{c}
\lstinputlisting{code/lists2.py}
\end{tabular}
\end{center}

\end{frame}

\begin{frame}
\frametitle{Lists}
Python provides many helpful built-in functions:

\begin{center}
\begin{tabular}{c}
\lstinputlisting{code/lists3.py}
\end{tabular}
\end{center}

\end{frame}

\begin{frame}
\frametitle{Lists}

Lists do not have to be homogeneous and they can be nested:

\begin{center}
\begin{tabular}{c}
\lstinputlisting{code/lists4.py}
\end{tabular}
\end{center}

\end{frame}


\begin{frame}
\frametitle{Lists}

The \texttt{for} and \texttt{in} operators can be used to iterate over and find elements in a list:

\begin{center}
\begin{tabular}{c}
\lstinputlisting{code/lists5.py}
\end{tabular}
\end{center}

\end{frame}


\begin{frame}
\frametitle{Lists}

List comprehensions form new lists by manipulating old ones

\begin{center}
\begin{tabular}{c}
\lstinputlisting{code/lists6.py}
\end{tabular}
\end{center}

\end{frame}


\begin{frame}
\frametitle{Tuples}

Tuples are similar to lists but they are immutable (cannot be modified)

\vspace{0.1in}

You can use tuples to enforce structure in your code
\end{frame}

\begin{frame}
\frametitle{Tuples}

\begin{center}
\begin{tabular}{c}
\lstinputlisting{code/tuples1.py}
\end{tabular}
\end{center}

\end{frame}

\section{Strings}

\begin{frame}
\frametitle{Strings}
Properties of strings are similar to those of lists (indexing, slicing)

\vspace{0.1in}

There are many built-in string commands that make string manipulations easy (we are already saw arithmetic on strings)

\end{frame}


\begin{frame}
\frametitle{Strings}

\begin{center}
\begin{tabular}{c}
\lstinputlisting{code/strings1.py}
\end{tabular}
\end{center}

\end{frame}

\begin{frame}
\frametitle{Strings}
Parsing a vector:
\begin{center}
\begin{tabular}{c}
\lstinputlisting{code/strings2.py}
\end{tabular}
\end{center}

\end{frame}

\begin{frame}
\frametitle{Strings}
Parsing a vector (one-liner):
\begin{center}
\begin{tabular}{c}
\lstinputlisting{code/strings3.py}
\end{tabular}
\end{center}

\end{frame}


\section{Dictionaries}

\begin{frame}
\frametitle{Dictionaries}

Dictionaries are maps from a set of keys to a set of values.  Keys need to be immutable (not changing)

\vspace{0.2in}

Dictionaries are also called ``associative arrays"
\end{frame}


\begin{frame}
\frametitle{Dictionaries}

$K = \{$ keys $\}$, $V = \{$ values $\}$.  $D: K \rightarrow V$:

\huge{
\begin{center}
$k \stackrel{D}{\longmapsto} v_k \in V$
\end{center}
} \normalsize{}

\vspace{0.2in}

In Python, you form $D$ by a series of insertions of tuples $(k, v_k) \in K \times V$.  It is ``fast" to compute $D(k)$.

\end{frame}

\begin{frame}
\frametitle{Dictionaries}

Example:

\begin{center}
\begin{tabular}{c}
\lstinputlisting{code/dicts1.py}
\end{tabular}
\end{center}

\end{frame}


\begin{frame}
\frametitle{Dictionaries}

Example:

\begin{center}
\begin{tabular}{c}
\lstinputlisting{code/dicts2.py}
\end{tabular}
\end{center}

\end{frame}


\begin{frame}
\frametitle{White space}

Python uses indentation (``white space") to group statements

\vspace{0.15in}

Each code block is indented the same amount (for loop, while loop, function definition, etc.)

\vspace{0.15in}

Python will complain if your indentation is incorrect
\end{frame}

\begin{frame}
\frametitle{White space}

\begin{center}
\begin{tabular}{c}
\lstinputlisting{code/white1.py}
\end{tabular}
\end{center}

\end{frame}


\begin{frame}
\frametitle{White space}

\begin{center}
\begin{tabular}{c}
\lstinputlisting{code/white2.py}
\end{tabular}
\end{center}

2-space or 4-space indentation is standard.

\end{frame}

\begin{frame}
\frametitle{White space}

Some people like this indentation structure and some people do not.

\vspace{0.2in}

... Python isn't going to stop using it

\end{frame}

\section{Functions}

\begin{frame}
\frametitle{More functions}

Functions:
\begin{itemize}
\setlength{\itemsep}{0.15in}
\item{Last time: functions are used to organize programs into coherent pieces}
\item{Today:
\begin{itemize}
  \setlength{\itemsep}{0.1in}
  \item{Specifically learn Python functions}
  \item{What is a lambda?}
\end{itemize}
}
\end{itemize}

\end{frame}

\begin{frame}
\frametitle{FFT}
Def is used to define a function

\begin{center}
\begin{tabular}{c}
\lstinputlisting{code/func1.py}
\end{tabular}
\end{center}

$x$ and $k$ are the arguments to the function
\end{frame}

\begin{frame}
\frametitle{FFT}

We can provide default argument values:

\begin{center}
\begin{tabular}{c}
\lstinputlisting{code/func2.py}
\end{tabular}
\end{center}

\end{frame}


\begin{frame}
\frametitle{FFT}

\begin{center}
\begin{tabular}{c}
\lstinputlisting{code/func3.py}
\end{tabular}
\end{center}

\end{frame}

\begin{frame}
\frametitle{Functions}
In Python, we can pass functions as objects:

\begin{center}
\begin{tabular}{c}
\lstinputlisting{code/lambda1.py}
\end{tabular}
\end{center}

\end{frame}

\begin{frame}
\frametitle{Lambdas}
Sometimes it is more convenient to not have to declare the functions:

\begin{center}
\begin{tabular}{c}
\lstinputlisting{code/lambda2.py}
\end{tabular}
\end{center}

These in-line function definitions are called lambdas

\end{frame}

\end{document}

%
\documentclass{beamer}
\usepackage{caption}
\usepackage{subcaption}
\usepackage{../../arbenson-math}

\usetheme{boxes}
\usecolortheme{seahorse}

\AtBeginSection[]
{
  \begin{frame}<beamer>
    \frametitle{\thesection}
    \tableofcontents[currentsection]
  \end{frame}
}

\title{CME 193: Introduction to Scientific Python \\
Lecture 4: NumPy and SciPy}
\author{Austin Benson \\
\vspace{0.1in}
Dan Frank \\
\vspace{0.1in}
Institute for Computational and Mathematical Engineering (ICME)}
\date{January 22, 2013}
\begin{document}

\maketitle

\section{NumPy}

\begin{frame}
\frametitle{What is NumPy?}

Wikipedia: NumPy is an extension to the Python programming language, adding support for large, multi-dimensional arrays and matrices, along with a large library of high-level mathematical functions to operate on these arrays.

\begin{itemize}
\setlength{\itemsep}{0.1in}
\item{At the core of the NumPy package, is the \textit{ndarray} object which encapsulates n-dimensional arrays of homogeneous data. 
}
\item{Many operations performed using \textit{ndarray} objects execute in compiled code for performance}
\item{The standard mathematical and scientific packages in Python use NumPy arrays}
\end{itemize}
\end{frame}

\begin{frame}
\frametitle{Array Creation}
Several ways to create arrays...
\lstset{basicstyle=\scriptsize}
\codeblock{code/array_creation.py}
\end{frame}

\begin{frame}
\frametitle{Array IO}
\lstset{basicstyle=\small}
\codeblock{code/array_io.py}
Other options control data types, delimiters, comments, headers, etc.
See documentation, especially "See Also".
\end{frame}

\begin{frame}
\frametitle{Array Attribtues}
Arrays are objects and so have attributes and methods.
\codeblock{code/array_attributes.py}
And many others. Explore in documentation or with TAB complete in ipython.
\end{frame}

\begin{frame}
\frametitle{Array Operations \& ufuncs}
Default behavior is elementwise 
\lstset{basicstyle=\scriptsize}
\codeblock{code/array_operations.py}
\end{frame}

\begin{frame}
\frametitle{Array Slicing and Indexing}
Similar to lists but a few new ways to select
\lstset{basicstyle=\scriptsize}
\codeblock{code/array_slice.py}
Boolean indexing can be very powerful as we will see in the exercises.
\end{frame}

\begin{frame}
\frametitle{Array Broadcasting \& Vectorization}
Broadcasting allows us to operate on arrays of different shapes by 'copying' smaller arrays when possible. This allows us to write more efficient and readable code (with fewer for loops).
\codeblock{code/array_broadcasting.py}
\end{frame}


\section{SciPy}

\begin{frame}
\frametitle{What is SciPy?}
SciPy is a library of algorithms and mathematical tools built to work with NumPy arrays.
\vspace{0.2in}
\begin{itemize}
\setlength{\itemsep}{0.1in}
\item{statistics - \textit{scipy.stats}}
\item{optimization - \textit{scipy.optimize}}
\item{sparse matrices - \textit{scipy.sparse}}
\item{signal processing - \textit{scipy.signal}}
\item{etc.}
\end{itemize}
\end{frame}


\begin{frame}
\frametitle{Example: KS-test}
Question: do two data samples come from the same distribution?
\lstset{basicstyle=\scriptsize}
\codeblock{code/ks_test.py}
\end{frame}

\begin{frame}
\lstset{basicstyle=\scriptsize}
\frametitle{Example: bootstrapped confidence interval}
\codeblock{code/bootstrap_mean.py}
\end{frame}

%\begin{frame}
%\frametitle{Example: k-means clustering}
%\codeblock{code/k_means.py}
%\end{frame}

\end{document}

%
\documentclass{article}
\newcommand{\assgnnum}{4}
\newcommand{\assigndate}{January 22}

\usepackage{amsmath}
%\usepackage{fullpage}
\usepackage{amssymb}
%\usepackage{bbm}
\usepackage{fancyhdr}
%\usepackage{paralist}
\usepackage{graphicx}
\usepackage[pdftex,colorlinks=true, urlcolor = blue]{hyperref}
\usepackage{../../arbenson-math}

\oddsidemargin 0in \evensidemargin 0in
\topmargin -0.5in \headheight 0.25in \headsep 0.25in
\textwidth 6.5in \textheight 9in
\parskip 6pt \parindent 0in \footskip 20pt

% set the header up
\fancyhead{}
\fancyhead[L]{CME193: In-class exercises \assgnnum}
\fancyhead[R]{\assigndate}
%%%%%%%%%%%%%%%%%%%%%%%%%%
\renewcommand\headrulewidth{0.4pt}
\setlength\headheight{15pt}


\newcommand{\p}{\ensuremath{\mathbf{P}}}
\renewcommand{\Pr}[1]{\ensuremath{\p \left \{ #1 \right \}}}
\newcommand{\nti}{\ensuremath{n \to \infty}}
\newcommand{\I}{\ensuremath{\operatorname{I}}}
\newcommand{\One}[1]{\ensuremath{\mathbbm{1}_{\left \{ #1 \right \}}}}
\newcommand{\E}{\ensuremath{\mathbf{E}}}
\newcommand{\Ex}[2][]{\ensuremath{\E_{#1} \left[ #2 \right]}}
\newcommand{\var}{\ensuremath{\operatorname{Var}}}
\newcommand{\cov}{\ensuremath{\operatorname{Cov}}}
\newcommand{\F}{\ensuremath{\mathcal{F}}}
\newcommand{\R}{\ensuremath{\mathbb{R}}}
\newcommand{\C}{\ensuremath{\mathbb{C}}}
\newcommand{\NormRV}[2]{\ensuremath{\operatorname{N}\left(#1, #2\right)}}
\newcommand{\BetaRV}[2]{\ensuremath{\operatorname{Beta}\left(#1, #2\right)}}
\newcommand{\argmax}{\operatornamewithlimits{argmax}}
\newcommand{\x}{\mathbf{x}}
\newcommand{\A}{\mathbf{A}}
\newcommand{\bb}{\mathbf{b}}


\newcounter{points}
\setcounter{points}{0}

\newcommand\setpoints[1]{\addtocounter{points}{#1}(#1 points)}
\newcommand\printpoints{Total number of points: \thepoints}

\newcommand{\eqD}{\ensuremath{\overset{\mathcal{D}}{=}}}

\setlength{\parindent}{0in}

\begin{document}

\pagestyle{fancy}
%\vspace*{15pt}
\begin{enumerate}

\item \textbf{Array Creation and Operations}
\begin{enumerate}
\item 
Create the following array
$\begin{bmatrix}
1 & 1 & 1 & 1 & 1\\
1 & 2 & 1 & 1 & 1\\
1 & 1 & 3 & 1 & 1\\
1 & 1 & 1 & 4 & 1
\end{bmatrix}$
\begin{verbatim}
A = np.ones((4,5))
A[np.arange(1,4), np.arange(1, 4)] = np.arange(2,5)
\end{verbatim}
\item  Compute the row sums of the above matrix 

\texttt{A.sum(axis=0)}

\item Compute the column sums of the above matrix

\texttt{A.sum(axis=1)}

\item
Download and read into memory the matrix found below. Check that it is equal to the array you created above. 
\begin{center}
\url{http://stanford.edu/~arbenson/cme193/data/lec4_array.txt}
\end{center}
\end{enumerate}

\begin{verbatim}
import os
os.system("wget http://stanford.edu/~arbenson/cme193/data/lec4_array.txt")
B = np.loadtxt('lec4_array.txt', skiprows=1, comments = '%', delimiter=',') 
np.all(B == A)
\end{verbatim}

\item{\textbf{Array Slicing and Indexing}}

Using the array above return the second and third rows and the columns containing an even number as a $2 \times  2$ array using... 
\begin{enumerate}
%\setcounter{enumii}{1}
\item integer indexes

\texttt{A[[1, 2],...][..., [1, 3]]}

\item slices

\texttt{A[1:3, 1:4:2]}

\item boolean arrays
\begin{verbatim}
A[np.array([False, True, True, False]), ...][...,
  np.array([False, True, False, True, False)]
\end{verbatim}


\item boolean arrays computed from the array


\begin{verbatim}
ind = np.apply_along_axis(lambda x: np.any(x % 2 == 0), 0, A)
A[1:3, ind]
\end{verbatim}
\end{enumerate}


\item \textbf{Broadcasting}

Using the above array assigned as \textit{arr}, describe the following operations


\begin{enumerate}


\item 
\begin{verbatim}
arr * 5.
\end{verbatim}
Multiplies ever element in arr by 5

\item 
\begin{verbatim}
arr * np.arange(arr.shape[1])
\end{verbatim}

Scales the columns of arr by 0, 1, 2, 3, 4
respectively\item 
\begin{verbatim}
arr * np.arange(arr.shape[0])
\end{verbatim}

Error, operation does not broadcast
\item 
\begin{verbatim}
arr.T * np.arange(arr.shape[0])
\end{verbatim}
Scales the rows of arr by 0, 1, 2, 3 respectively and returns the transpose of arr scaled in this way.

\item compute the dot product of the array with $\begin{bmatrix}
0\\
1\\
2\\
3\\
4\\
\end{bmatrix}$
in two ways

\texttt{np.sum(arr * np.arange(5), axis=1)}

\texttt{np.dot(arr, np.arange(5)}
\end{enumerate}
%\printpoints.

\end{enumerate}
\end{document} 
